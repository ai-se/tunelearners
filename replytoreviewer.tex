\newcommand{\bi}{\begin{itemize}[leftmargin=0.4cm]}
\newcommand{\ei}{\end{itemize}}
\newcommand{\be}{\begin{enumerate}}
\newcommand{\ee}{\end{enumerate}}
\newcommand{\tion}[1]{\S\ref{sect:#1}}
\newcommand{\fig}[1]{Figure~\ref{fig:#1}}
\newcommand{\tab}[1]{Table~\ref{tab:#1}}
\newcommand{\eq}[1]{Equation~\ref{eq:#1}}
\documentclass[final,twocolumn,5p]{elsarticle}
\begin{document}
\clearpage
\pagenumbering{roman}
\setcounter{page}{1} 
\section*{Reply to Reviews}

Thank your for your comments. The typos listed by reviewers
have been fixed and the remainder of the paper has been given
a careful proof read.

The reviewers raised certain issues which we respond  to as follows.-

{\it I am concerned about the reproduction of work from elsewhere. Although the relevant paper and book are cited in section 2.3 ([27, 28]) the amount of material that is reproduced is surprisingly large. I am also unsure what "presenting some new results from Rahman et al. [28]" actually means. It would obviously be wrong to present other researchers' results as if newly discovered but I doubt that this is the intention. Indeed, on checking the other paper, there is no obvious overlap. This needs to be clarified.}

Apologies to the reviewer for our unclear text that suggests
that this paper inappropriately copied  content from
elsewhere. This is not the case since nearly all of this paper has not been submitted
or published to another venue. The exception to that is:
\begin{itemize}
\item Section 2.3 and Table1 contains 1 page of tutorial material which we have adapted from other papers.
e.g. as the reviewer correctly points out,
the  'Easy to use' paragraph is taken nearly verbatim from [28], as are the 'widely-used', and 'useful' paragraphs.
\end{itemize}
Note that apart from Section 2.3,  10 of the 11 pages of this text are   completely new.

As to the reference to " presenting some new results from Rahman et al. [28]", that refers to one paragraph (120 words) end of section 2.3. 
Please note that we have removed the misleading text that raised that confusion
(start of 2.3).

{\em I think the way that the results are presented does not do justice to the findings in the abstract (that the improvements are large, and the tuning is simple). In particular, tables 8 and 9 are not very clear; why show the naive column?}

Thank for your that comment- we have simplified that presentation by separating the tuning \#evaluations from the runtimes-- see Table 8 and the new Table 9.

{\em I am distracted by the frequent use of footnotes. If the material is important and worthy of mention then it should be included in the main body of the paper. However, a reference to the relevant work may be sufficient and is sometimes preferable to using a footnote.}

Quite true- those footnotes are needlessly distracted.
They have now been incorporated into the text.



{\em  Eq 1 - what are d and T? - please use a where clause}



That where clause is now added after Equation 1. That clause says

\begin{quote} ... where  $d_i$ is the number of observed issues and $T$
is some threshold defined by an engineering judgement; we use $T=1$.\end{quote}

\end{document}

