% \documentclass{sig-alternative}
\documentclass[conference]{IEEEtran}
\usepackage{multirow}
\usepackage{color}
\usepackage{graphics} 
\usepackage{cite}
 
\usepackage{rotating}
\usepackage{eqparbox}
\usepackage{graphics}
\usepackage{colortbl} 
\usepackage{balance}
\usepackage{picture}
\usepackage{algorithm}
\usepackage{algorithmicx}
\usepackage{algpseudocode}
\renewcommand{\footnotesize}{\scriptsize}
\definecolor{lightgray}{gray}{0.8}
\definecolor{darkgray}{gray}{0.6}
\renewcommand{\algorithmicrequire}{\textbf{Input:}}
\renewcommand{\algorithmicensure}{\textbf{Output:}}
%%% graph
\newcommand{\crule}[3][darkgray]{\textcolor{#1}{\rule{#2}{#3}}}
%\newcommand{\rone}{\crule{1mm}{1.95mm}}
%\newcommand{\rtwo}{\crule{1mm}{1.95mm}\hspace{0.3pt}\crule{1mm}{1.95mm}}
%\newcommand{\rthree}{\crule{1mm}{1.95mm}\hspace{0.3pt}\crule{1mm}{1.95mm}\hspace{0.3pt}\crule{1mm}{1.95mm}}
%\newcommand{\rfour}{\crule{1mm}{1.95mm}\hspace{0.3pt}\crule{1mm}{1.95mm}\hspace{0.3pt}\crule{1mm}{1.95mm}\hspace{0.3pt}\crule{1mm}{1.95mm}} 
%\newcommand{\rfive}{\crule{1mm}{1.95mm}\hspace{0.3pt}\crule{1mm}{1.95mm}\hspace{0.3pt}\crule{1mm}{1.95mm}\hspace{0.3pt}\crule{1mm}{1.95mm}}
\newcommand{\quart}[3]{\begin{picture}(100,6)%1
{\color{black}\put(#3,3){\circle*{4}}\put(#1,3){\line(1,0){#2}}}\end{picture}}
\definecolor{Gray}{gray}{0.95}
\definecolor{LightGray}{gray}{0.975}


\newcommand{\rone}{}
\newcommand{\rtwo}{}
\newcommand{\rthree}{}
\newcommand{\rfour}{} 
\newcommand{\rfive}{}


\newcommand{\wei}[1]{\textcolor{red}{Wei: #1}} 

%% timm tricks
\newcommand{\bi}{\begin{itemize}[leftmargin=0.4cm]}
\newcommand{\ei}{\end{itemize}}
\newcommand{\be}{\begin{enumerate}}
\newcommand{\ee}{\end{enumerate}}
\newcommand{\tion}[1]{\S\ref{sect:#1}}
\newcommand{\fig}[1]{Figure~\ref{fig:#1}}
\newcommand{\eq}[1]{Equation~\ref{eq:#1}}

%% space saving measures

\usepackage[shortlabels]{enumitem}  

\usepackage{url}
\def\baselinestretch{1}


\setlist{nosep}
 \usepackage[font={small}]{caption, subfig}
\setlength{\abovecaptionskip}{1ex}
 \setlength{\belowcaptionskip}{1ex}

 \setlength{\floatsep}{1ex}
 \setlength{\textfloatsep}{1ex}
 \newcommand{\subparagraph}{}

\usepackage[compact,small]{titlesec}
\DeclareMathSizes{7}{7}{7}{7} 
\setlength{\columnsep}{7mm}

\begin{document}
% \conferenceinfo{FSE}{'15 Bergamo, Italy}
\title{ Analytics Without  Tuning Considered Harmful?}
% \numberofauthors{1}
\author{\alignauthor Wei Fu, Tim Menzies, Xipeng Shen \\
       \affaddr{Computer Science, North Carolina State University, Raleigh, USA}\\
       fuwei.ee, tim.menzies, xipengshen@gmail.com}
\maketitle
\pagestype{plain}
\begin{abstract}
One of the ``black arts'' of data mining is setting the tuning
parameters that control  the miner.  We offer a simple,
automatic, and very effective  method for finding those tunings.
For the purposes of learning
software defect predictors this  methid can quickly
find  good tunings that  dramatically change   the performance of a learner.
For example,
in this paper we show   tunings that  alter detection  precision  
 from 2\% to 98\%.

These results prompt for a change to standard methods in software analytics.
At least for defect prediction, 
it is no longer enough to just run a data miner and present the result
{\em without}  conducting a tuning optimization study.
The implications for other kinds of  analytics is now  an open and pressing issue.

%RQ1: Does tuning affect learners' performance?
%RQ2: How to choose 

\end{abstract}

% A category with the (minimum) three required fields
\vspace{1mm}
\noindent
{\bf Categories/Subject Descriptors:} 
D.2.8 [Software Engineering]: Product metrics;
I.2.6 [Artificial Intelligence]: Induction

 
\vspace{1mm}
\noindent
{\bf Keywords:} defect prediction, CART, random forests,
differential evolution,
search-based software engineering.

\section{Introduction}
 
In the $21^{st}$ century, it is now impossible
to manually browse all the available software project
data. The PROMISE repository of SE data has grown to 200+ projects~\cite{promise15}
and this is just one of over a dozen open-source repositories
that are readily available to researchers~\cite{rod12}.
For example,  the time of this writing (Feb  2015), our web searches show that Mozilla Firefox has over 1.1 million bug reports, and platforms such as GitHub host over 14 million projects. 



 
Faced with this information overload,
researchers in empirical SE
use  data miners  to generate 
defect predictors from static code measures.
Such   measures can be
automatically extracted from the code base, with very little effort
even for very large software systems~\cite{nagappan05}. 

One of the ``black arts'' of data mining is setting the tuning
parameters that control  the choices within that data miner.
Prior to this work, our intuition was that tuning would change the behavior or a data miner, to some degree. Nevertheless, we rarely tuned our  defect predictors 
since we reasoned
that a data miner's default tunings have been well-explored by the developers of those algorithms (in which case
tuning would not lead to large performance improvements).
Also, we suspected that
tuning would take so long and be so CPU intensive that the benefits gained   would not be worth effort.

The results of this paper show that neither of these points is supportable, at least when learning
defect detectors from static code attributes since:
\bi
\item
Tuning static code defect predictors is {\em remarkably simple} and can {\em dramatically improve the performance}. 
\ei
Prior to these experiments, our expectation was tuning
would be an  extensive and expensive evolutionary optimization procedure. 
A standard run of such evolutionary optimizers requires   thousands,
if not millions, of evaluations.
To
our surprise, we achieved dramatic improvements in the performance scores
of our data miners after  mere 50 to 80 evaluations (!!) of a very simple evolutionary 
optimizer  called differential evolution~\cite{storn1997differential}.
Better yet,  those  tunings (found so quickly)   
  have a dramatic change to the performance of a learner. For example,
in this paper we show that tuning can alter the precision of
a software defect predictor from 2\% to 98\%.

These results challenge much prior work. 
Firstly, there exist research papers
that use data miners to   show that certain factors
are more influential than others for (say)
predicting defects. As shown below, such conclusions can be dramatically
changed by the tuning process since those  ``influential'' factors are very different pre- and post- tuning. Also, those factors tend to  change from project to project or if the goal
of the tuning is altered.
Hence, many old papers    need to be revisted  and perhaps revised~\cite{bell2013limited,rahman2013how,me02k,moser2008comparative,zimmermann2007predicting,herzig2013predicting}.  
For example, one of us (Menzies) used data miners
to assert that some factors were more important than others for predicting
successful software reuse~\cite{me02k}. That assertion should now be doubted since that
Menzies study did not conduct a tuning study before reporting what factors the data miners
found where most influential.

Secondly, several  prominent IEEE TSE papers~\cite{lessmann2008benchmarking,hall11,me07b} have claimed 
that learnerX is better than learnerY for some software analytics task.
For example, a recent IEEE TSE article claimed that the 
CART decision tree learner was far worse than Random Forests for
software defect prediction~\cite{lessmann2008benchmarking}. 
Such conclusions do not survive tuning.
For example,
after tuning, the worst untuned learner (CART) can out-perform the supposedly
best learner (Random Forests). Hence, all those prior results that ranked learners for software
analytics now need to be revisited and perhaps revised.

Thirdly, it is standard practice to use the default ``off-the-shelf'' tunings  for data mining tools (previously
we have defended that approach, arguing that it
encourages reproducibility~\cite{me15:book1}). That ``off-the-shelf''  policy
can no longer be condoned. For example, suppose our default
number of trees in  a Random Forest was   $F=100$ (which is the default in our implementation). 
After tuning, we see that our optimizer often used $F$ values that were nowhere near that default:

% {\scriptsize
% \[F \in \left\{\begin{array}{l} 55,  65, 70,   82, 88, 96, 100,  102,  104, 107,\\
%                                 108,  119, 133,  140, 140,   147,  145,  142   \end{array}\right\}
% \]}


% {\scriptsize
% \[F \in \left\{\begin{array}{l} 50, 59, 63, 63, 69, 71, 81, 85,85, 88,\\
%                                 92, 95, 96, 99, 99,  100, 104, 106, 107,\\
%                                 109, 111, 112, 121, 121, 123, 125, 128,\\
%                                 131, 132, 135, 140, 141, 141, 150   \end{array}\right\}
% \]}



{\scriptsize
\[F \in \left\{\begin{array}{l} 50, 58, 59, 63, 63, 66, 67, 73, 74, 74, \\
                                75, 77, 80, 82, 85, 96, 96, 97, 97, 101, \\
                                103, 107, 111, 111, 112, 120, 125, 125,  \\
                                130,130, 138, 144, 149, 150   \end{array}\right\}
\]}

In summary, it is now an open and pressing research issue to check if
analytics without parameter tuning is considered {\em harmful} or, at the 
very least, {\em misleading}.
Clearly, we  now doubt  conclusions based on
``off-the-shelf'' tunings.
Further,
it is no longer enough to just run a data miner and report the result
{\em without} first conducting an tuning optimizations study.

Before beginning, we digress for one  clarification.
This paper is {\bf not} arguing that
software analytics is somehow wrong-headed or misguided.
In the age of the Internet and global access to software engineering data,
there exists the  problem of information overload. {\em Something} must be done to
allow analysts to make conclusions via an automatic analysis over a lot of data.
The results of this paper is that for a particular local context
(a specific data set and a specific goal) there exists  
methods for optimizing the conclusions reached in that context.  
Those conclusions
may not generalize to other contexts but this  is not a council for despair. 
As shown here, 
there  exists general methods for finding
local conclusions in a particular context. Further,
those
methods are  very simple to implement and very fast to execute.


\section{Motivating Example}\label{sect:eg}

This section is  a  demonstration
of the impact of tuning.


Linear  regression fits a  line to a set of points in order to minimize  predictive
error and maximizes the correlation\footnote{
{\em Correlation}
measures how closely two variables co-vary.  It ranges from   from -1
(perfect negative correlation) through 0 (no correlation) to +1
 (perfect positive correlation).
Let  $a_i$ and $p_i$ denote some actual and predicted values respectively; and  $n$ and $\overline{x}$ denote
 the number of observations and the mean of the $n$ observations, 
respectively. Then:
$S_{\mathit{PA}}=(\sum_i (p_i - \overline{p})(a_i -
  \overline{a}))/(n-1)$ and
$S_{p}=(\sum_i (p_i - \overline{p})^2)(n-1)$
and 
$S_{a}=(\sum_i (a_i - \overline{a})^2)/(n-1)$ and
correlation is  $c=S_{\mathit{PA}}/\sqrt{S_pS_a}$.}  denoted as $c$.
Suppose  a researcher wants to use linear regression
to test if Halstead's~\cite{halstead77} measures
of   function complexity
(number of symbols programmers has to understand) are   {\em better than}
mere lines of code for predicting
software defects.  That researcher might believe that Halstead's cognitive approach to
software bugs is better suited to code refactoring tools since it offers 
more ways to alter functions that just some coarse grain lines of code measure.


To test that belief, our  researcher applies regression to  some defect logs.
 Here are two equations (learned from the NASA data at goo.gl/pGDfvp)
that use just lines of code or the Halstead measures $N,V,L,D,I,E,B,T$ seen in a
software module (in this case, a  function).
Note that the Halstead correlation $c_2$
is worse than the $c_1$ correlation  from lines of code-- which suggests
 our researcher should not use Halstead .

{\scriptsize \[
\begin{array}{l|l|ll}
\mathit{measures} & d= \mathit{\#defects} & \mathit{correlation}\\\hline
\mathit{LOC}   &d_1= 0.0164 +0.0114\mathit{LOC}\ & c_1 = 0.65\\\hline
\mathit{Halstead} & d_2= 0.231 + 0.00344N  +  0.0009V    \\    
                 &   - 0.185L- 0.0343D      - 0.00541I  \\ 
                 & + 0.000019E + 0.711B  - 0.00047T  & c_2=-0.36  
\end{array}
\]
}
 

\noindent
We now explore how tuning can change the above  conclusion. Suppose the  predictors $d_1$ and $d_2$  learned from LOC or Halstead
are used to call an inspection
team to check for errors in   parts of the code using:
\begin{equation}\label{eq:yesno}\scriptsize
\mathit{inspect}= \left\{
\begin{array}{ll}
d_i \ge T \rightarrow \mathit{Yes}\\
d_i <   T \rightarrow \mathit{No} 
\end{array}\right.
\end{equation}
\fig{pd1} shows the effects of tuning. Not surprisingly,
at $T=0$, all modules get inspected so the false alarm rate is very high. To reduce that
problem, we can increase $T$:   the false alarm rate falls below
20\% at $T=0.45$ (for Halstead). 

 

\begin{figure}[!t] 

\renewcommand{\baselinestretch}{0.8}
{\scriptsize
\begin{center}

\% recall (probability of detection):   

\includegraphics[width=3in]{lsrvscostpd.pdf}

\% false alarms:

\includegraphics[width=3in]{lsrvscostpf.pdf}
\end{center}}
\caption{
 Y-axis shows probability of false alarm,
  and
  probability of recognizing defective modules  seen using \mbox{$d_i \ge T$}.
  Curves calculated from the KC2 dataset from the PROMISE repository goo.gl/pGDfvp.
 }\label{fig:pd1}
 \end{figure}
 
Note that either the Halstead or LOC detector can reach some desired
level of recall, regardless of their correlations, just by
selecting the appropriate threshold value. For example, in \fig{pd1}, see the recall=75\% values
found at {\em either} $d_i\ge 0.65$ or $d_2\ge 0.45$ (and at the threshold, the false alarm rates
were very similar: 14\% and 19\%).

The  point here is that   the true value of a detector
could not be assessed {\em without} conducting a  tuning study in the context of some business case (in this case, 
issuing a request to an inspection team to review some module).  
Hence, it is important to explore tuning
so the rest of this paper repeats the analysis of this section, but for 
more complex learners.


\section{Background Notes}
 

\subsection{Defect Prediction}


This section is our standard introduction to defect prediction~\cite{me15:book1},
plus    new results from Rahman et al.~\cite{rahman14:icse}. 
 


\begin{figure*}[!t]
\renewcommand{\baselinestretch}{0.8}\begin{center}
{\scriptsize
\begin{tabular}{c|l|p{4.7in}}
amc & average method complexity & e.g. number of JAVA byte codes\\\hline
avg\_cc & average McCabe & average McCabe's cyclomatic complexity seen
in class\\\hline
ca & afferent couplings & how many other classes use the specific
class. \\\hline
cam & cohesion amongst classes & summation of number of different
types of method parameters in every method divided by a multiplication
of number of different method parameter types in whole class and
number of methods. \\\hline
cbm &coupling between methods &  total number of new/redefined methods
to which all the inherited methods are coupled\\\hline
cbo & coupling between objects & increased when the methods of one
class access services of another.\\\hline
ce & efferent couplings & how many other classes is used by the
specific class. \\\hline
dam & data access & ratio of the number of private (protected)
attributes to the total number of attributes\\\hline
dit & depth of inheritance tree &\\\hline
ic & inheritance coupling &  number of parent classes to which a given
class is coupled (includes counts of methods and variables inherited)
\\\hline
lcom & lack of cohesion in methods &number of pairs of methods that do
not share a reference to an instance variable.\\\hline
locm3 & another lack of cohesion measure & if $m,a$ are  the number of
$methods,attributes$
in a class number and $\mu(a)$  is the number of methods accessing an
attribute, 
then
$lcom3=((\frac{1}{a} \sum_j^a \mu(a_j)) - m)/ (1-m)$.
\\\hline
loc & lines of code &\\\hline
max\_cc & maximum McCabe & maximum McCabe's cyclomatic complexity seen
in class\\\hline
mfa & functional abstraction & number of methods inherited by a class
plus number of methods accessible by member methods of the
class\\\hline
moa &  aggregation &  count of the number of data declarations (class
fields) whose types are user defined classes\\\hline
noc &  number of children &\\\hline
npm & number of public methods & \\\hline
rfc & response for a class &number of  methods invoked in response to
a message to the object.\\\hline
wmc & weighted methods per class &\\\hline
\rowcolor{lightgray}
defect & defect & Boolean: where defects found in post-release bug-tracking systems.
\end{tabular}
}
\end{center}
\caption{OO measures used in our defect data sets.}\label{fig:ck}
\end{figure*}



Human programmers are clever, but flawed. Coding  adds functionality, but also defects.
Hence, software sometimes crashes (perhaps at the most awkward or dangerous moment) or delivers
the wrong functionality. For a very long list of software-related errors,
see  Peter Neumann's ``Risk Digest'' at catless.ncl.ac.uk/Risks.

Since programming inherently
introduces defects into  programs, it is important to test them before they're used.
Testing is expensive.
Software assessment budgets are finite
while assessment effectiveness increases 
exponentially with assessment effort.
For example, for  black-box testing methods,
a {\em linear} increase
in the confidence $C$ of finding  defects
can take {\em exponentially} more effort\footnote{A randomly selected 
input to a program will find a fault with probability $p$.
After $N$ random black-box tests, the chances of the inputs 
not revealing any fault 
is $(1-p)^N$. Hence, the chances $C$ of seeing the fault is $1-(1-p)^N$
which can be rearranged to 
 $N(C,p)=log(1 -
C)/log(1-p)$. For example, $N(0.90,10^{-3})=2301$
but $N(0.98,10^{-3})=3901$; i.e. nearly double the number of tests.}.
Exponential costs quickly exhaust finite resources so
standard practice is to apply the best
available  methods on code sections that seem   most critical. 
But 
any method that focuses on parts of the code
can blind us to defects in other areas. Some  {\em
lightweight sampling policy} should be used to explore the rest of the system.  This
sampling policy will always be incomplete.
Nevertheless, it is the only option when
resources prevent a complete assessment of everything.

One such lightweight sampling policy is defect predictors learned from static code attributes.
Given software described in the attributes of \tab{ck},   data miners can
learn where the probability of software defects is highest.

The rest of this section argues that such defect predictors are   {\em easy to
use}, {\em widely-used}, and {\em useful} to use.

{\em Easy to use:} Static code attributes can be automatically collected, even for very large systems~\cite{nagappan05}.
Other methods, like  manual code reviews, are far slower and far more labor-intensive.
For example, depending on the review methods, 8 to 20 LOC/minute can be
inspected and this effort repeats for all members of the review team,
which can be as large as four or six people~\cite{me02f}. 

{\em Widely used:}  Researchers and industrial practitioners  use static attributes to guide software 
quality predictions.
 Defect prediction models have been reported
  at Google~\cite{lewis13}.
Verification and validation (V\&V) textbooks
(\cite{rakitin01}) advise using static code complexity attributes
to decide which modules are worth manual inspections.  


{\em Useful:}
Defect predictors often  find the location of  70\% (or more)
of the defects in code~\cite{me07b}.
Defect predictors have some level of generality:
predictors learned at NASA~\cite{me07b} have also been found useful elsewhere
(e.g. in Turkey~\cite{tosun10,tosun09}.
The success of this method in  predictors in finding bugs is   markedly
higher than other currently-used
industrial
methods such as manual code reviews. For example, 
a  panel at {\em IEEE Metrics
2002}~\cite{shu02} concluded that manual software  reviews can find ${\approx}60\%$ 
of defects.
In other work, 
Raffo documents the typical    defect detection capability of
industrial review methods:   around 50\%
 for full Fagan inspections~\cite{fagan76} to
21\% for less-structured inspections.

Not only do static code defect predictors perform well compared to manual methods,
they also are competitive with certain automatic methods.
A recent study at ICSE'14, Rahman et al.~\cite{rahman14:icse} compared
(a) static code analysis tools FindBugs, Jlint, and Pmd and (b)
static code defect predictors
(which they called ``statistical defect prediction'') built using logistic regression.
They found  no significant differences in the cost-effectiveness
of these  approaches. Given this equivalence, it is significant to note that 
static code defect prediction can be quickly adapted to new languages by building lightweight
parsers that find   information like \tab{ck}. The same is not true for   static code analyzers-- these need  extensive modification before they can be used on new
languages.



 

\subsection{Data Mining Algorithms}
 
This section describes this study's learners (CART~\cite{brieman00}, Random Forest~\cite{breiman84}, 
and WHERE~\cite{menzies2013local}) and their
tuning parameters (summarised in \fig{parameters}).

Our implementations
for CART and Random Forest comes from 
SciKitLearn~\cite{scikit-learn}.
WHERE is available from
github.com/ai-se/where\footnote{\wei{ need a nice  where repo. with an
data and code examples sub-directory. clean code. throw out anything not needed. before april1.}}.
WHERE has the most tuning parameters. This turns out
to be important since, in the experiments shown below, tuned WHERE 
out-performed the other learners-- a result suggesting that
if you are going to tune learners, then use one with many tuning options.

%%%%%%%%%%%%%%%% list of parameters%%%%%%%%%%%%%%%%%%%%%
\renewcommand\arraystretch{1.2}
\begin{figure*}[t!]
\scriptsize
  \centering
	\begin{tabular}{|c|c|c|c|l|}
	\cline{1-5}
	\begin{tabular}[c]{@{}c@{}}Learner \\ Name\end{tabular} & Parameters & Default &\begin{tabular}[c]{@{}c@{}}Tuning\\ Range\end{tabular}& 
\multicolumn{1}{c|}{Description} \\ \hline
	\multirow{8}{*}{\begin{tabular}[c]{@{}c@{}}Where-based\\ Learner\end{tabular}} 
	& threshold & 0.5 &[0.01,1]& The value to determine defective or not .\\ \cline{2-5} 
	& infoPrune & 0.33 &[0.01,1]& The percentage of features to consider  for the best 
split to build CART tree\footnote{Since the Where-based learner will build two trees, the first 
one is for clustering and the second one is building prediction model. we explicitly call Where-
clustering tree and CART tree, respectively}. \\ \cline{2-5} 
	 & min\_sample\_split & 4& [1,10]& The minimum number of samples required to split an internal node of
CART tree. \\ \cline{2-5} 
	 & min\_Size & 0.5 &[0.01,1]& \begin{tabular}[c]{@{}l@{}}The value to determine the minimum 
number of samples to be a Where-clustering tree \\ based on  ${n\_samples}^ {min\_Size}$.
\end{tabular} \\ \cline{2-5} 
    & wriggle & 0.2 &[0.01, 1] & The threshold to determine which branch in  Where tree to be pruned\\ \cline{2-5}
	 & depthMin & 2 & [1,6]&The minimum depth of the tree below which no pruning for Where-
clustering tree. \\ \cline{2-5} 
	 & depthMax & 10 &[1,20]& The maximum depth of the Where-clustering tree. \\ \cline{2-5} 
	 & wherePrune & False &T/F& Whether or not to prune the Where-clustering tree. \\ \cline{2-5}
	 & treePrune & True &T/F& Whether or not to prune the classification tree built by CART. \\ \cline{2-5} 
\hline
\multirow{4}{*}{CART} & threshold & 0.5 &[0,1]& The value to determine defective or not. \\ \cline{2-5} 
	 & max\_feature & None &[0.01,1]& The number of features to consider when looking for the best 
split. \\ \cline{2-5} 

	 & min\_sample\_split & 2 &[2,20]& The minimum number of samples required to split an 
internal node. \\ \cline{2-5} 
	 & min\_samples\_leaf & 1 & [1,20]&The minimum number of samples required to be at a leaf 
node. \\ \cline{1-5}  
       \multirow{5}{*}{\begin{tabular}[c]{@{}c@{}}Random \\ Forests\end{tabular}}  & threshold & 0.5 & [0.01,1] & The value to determine defective or not. \\ 
\cline{2-5} 
	 & max\_feature & None &[0.01,1]& The number of features to consider when looking for the best 
split. \\ \cline{2-5} 
	 & max\_leaf\_nodes & None &[1,50]& Grow trees with max\_leaf\_nodes in best-first fashion. \\ \cline{2-5} 
	 & min\_sample\_split & 2 &[2,20]& The minimum number of samples required to split an 
internal node. \\ \cline{2-5} 
	 & min\_samples\_leaf & 1 &[1,20]&The minimum number of samples required to be at a leaf 
node. \\ \cline{2-5} 
	 &  n\_estimators & 100 & [50,150]&The number of trees in the forest.\\ \cline{2-5}
	 \hline

	\end{tabular}
    \caption {List of parameters to be tuned.}
\label{fig:parameters}
\end{figure*}
 
\subsection{Why Study These Algorithms?}


This paper studies WHERE since this was the first learner we tried to tune and, as shown below,
it offers an interesting case study on the benefits of tuning.

This paper also studies  CART and Random Forest since  these were used in 
a recent IEEE TSE paper by Lessmann et al.~\cite{lessmann2008benchmarking} that compared 21 different 
learners for software defect prediction:
\bi
\item
{\em Statistical classifiers:}
Linear    discriminant analysis,
Quadratic discriminant analysis,
Logistic regression,
Naive Bayes,
Bayesian networks,
Least-angle regression,
Relevance vector machine,

\item
{\em Nearest neighbor methods:}
k-nearest neighbor,
K-Star

\item
{\em Neural networks:}
Multi-Layer Perceptron,
Radial bias function network,

\item
{\em Support vector machine-based classifiers:}
Support vector machine,
Lagrangian SVM
Least squares SVM,
Linear programming,
Voted perceptron,

\item
{\em Decision-tree approaches:}
C4.5 decision tree,
CART,
Alternating decision tree.
\item
{\em Ensemble methods:}
Random Forest,
Logistic Model Tree.
\ei
In that study, CART was severely trounced (was ranked last) and Random Forest was
the standout best method. The experiments shown below both confirm and refute
that ranking. In a result consistent with the prior result, untuned Random Forest performs best.
However, after tuning, the worst learner found by Lessmann et al. (CART) performed better
than Random Forest.
  

\subsection{Learners and Their Tunings}

CART, Random Forest, and WHERE are all  tree learners that divide a data set, then recurse
on each split.
If data contains more than {\em min sample split}, then a split is attempted.
On the other hand, if a split contains no more than {\em min samples leaf}, then recursion stops. For CART and Random Forest use a 
user-supplied constant for this parameter while
WHERE computes $m$={\em min samples leaf} from the size of the data
sets via  $m=\mathit{size}^\mathit{min size}$ (so, for WHERE,
{\em min size} is the parameter to be tuned).

These learners
generate numeric predictions which are converted
into binary ``yes/no'' decisions via \eq{yesno}. Hence, they all use the {\em threshold} value $T$ discussed in \tion{eg}.

These learners use different techniques to explore the splits:
\bi
\item
CART finds the attributes whose ranges contain rows with least variance in the number
of defects\footnote{If an attribute ranges $r_i$ is found in 
$n_i$ rows each with a  defect count variance of $v_i$, then CART seeks the attributes
whose ranges minimizes $\sum_i \left(\sqrt{v_i}\times n_i/(\sum_i n_i)\right)$.}.
\item
Random Forest    divides data like CART,
but it builds $F>1$  trees, each time with a subset of
the attributes (selected at random). 
\item
WHERE projects the data on to a dimension it synthesizes from the raw data using
a process analogous to principle component analysis\footnote{
PCA  synthesises  new
attributes $e_i, e_2,...$
that extends across the dimension of greatest  variance in the data  with attributes $d$.  
This process  combines
redundant  variables into a smaller set of variables  (so $e \ll d$) since those
redundancies become (approximately) parallel lines
in $e$ space. For all such redundancies \mbox{$i,j \in d$}, we 
can ignore $j$ 
since effects that change over $j$ also
change in the same way over $i$.
PCA is also useful for skipping over noisy variables from $d$-- these
variables are effectively ignored since    they  do not contribute to the variance in the data.}.
WHERE   divides  at the median point of that projection. On recursion,
this generates a dendogram, the leaves of which are clusters of  very similar examples.
\ei
WHERE's {\em infoPrune} tuning parameter then choices the
attributes   that best select  different clusters.
WHERE pretends its clusters are ``classes'', then 
asks the InfoGain of the
Fayyad-Irani discretizer~\cite{FayIra93Multi}, to rank the attriubutes.
WHERE then ignores everything except the top   {\em infoPrune} percent of the sorted
attributes.

Optionally, if the {\em where prune} option is set, 
WHERE  continues to applies infogain criteria  recursively to build a tree that selects for the
different clusters. If WHERE's {\em tree prune} parameter is enabled, then WHERE also prunes  superfluous sub-trees. For example, if a sub-tree and its parent have the same 
majority cluster
(one that occurs most frequently), then we prune the sub-tree.
This tree pruning  sometimes
prunes aways all  cluster selectors branches. To tame this effect, the {\em wriggle} parameter
blocks tree pruning for at least the first {\em wriggle} number of initial branches.

Other tuning parameters are learner specific. For example,
{\em max feature} is used by
CART and Random Forest to select the number of attributes
used to build one tree.
CART's default is to use all the attributes while 
Random Forest usually selects the square root of the number
of attributes.
Also,
  {\em max leaf nodes} is the upper bound on leaf notes generated in a 
  Random Forest.



\subsection{Tuning Algorithms}


 \subsubsection{Parametric Tuning Algorithms}
The  goal of this paper is to adjust the tuning parameters of \fig{parameters}
in order to   optimize (improve) some particular performance scores
generated by a particular learner being applied to  a particular data set.
For this task, we do not use traditional parametric numeric optimizer  
such as  gradient descent optimizers~\cite{saltelli00} that require models comprise
differential functions (i.e. functions of real-valued variables whose derivative exists at each point in its domain).
This is impractical  for  our learners since their internal states are   not a smoothly differential continuous function.
Rather, learners being tuned  contains many regions with many different properties (tuning options can
drive the learner into very different modes with very different performance properties).


 \subsubsection{Non-Parametric Tuning Algorithms}
 
Non-parametric  optimizers   make no assumption
about the model being only differential functions. One such optimizer
is simulated annealing. SA generates {\em new} solutions
 by randomly perturbing (a.k.a. ``mutating'') some part of an {\em old}
 solution.  {\em New} replaces {\em old} if (a) it scores higher; or
 (b) it reaches some probability set by a ``temperature'' constant. Initially,
 temperature is high so SA jumps to sub-optimal solutions (this allows
 the algorithm to escape from local minima). Subsequently, the
 ``temperature'' cools and SA only ever moves to better {\em new}
 solutions. 
 SA is often used in search-based SE
 e.g.~\cite{fea02a,me07f}, perhaps due to its simplicity.

SA was invented   in the 1950s, when
 computer RAM was very small~\cite{kirkpatrick83}. A standard SA algorithm needs
 only space for three solutions {\em new, old} and the {\em best} seen so far.
  In the 1960s, when more RAM became available, it became standard to
 generate many {\em new} mutants, and then combine together parts of
 promising solutions~\cite{goldberg79}.  Such {\em evolutionary
   algorithms} (EA) work in {\em generations} over a population of
 candidate solutions.  Initially, the population is created at random.
 Subsequently, each generation makes use of select+crossover+mutate
 operators to pick promising solutions, mix them up in some way, and
 then slightly adjust them.
 EAs are also often used in search-based
 software engineering, particularly in test case generation~\cite{andrews07,andrews10}
 or refactoring~\cite{Weimer:2009}

 Later work focused on creative ways to control the
 mutation process. Tabu search and scatter search
 work to bias new mutations away from prior
 mutations~\cite{Glover1986563,Beausoleil2006426,Molina05sspmo:a,4455350}.
 Particle swarm
 optimization randomly mutates multiple solutions
 (which are called ``particles''), but biases those
 mutations towards the best solution seen by one
 particle and/or by the neighborhood around that
 particle~\cite{pan08}.
 Differential evolution mutates solutions by
 interpolating between members of the current
 population~\cite{storn1997differential}.  
 
Another more recent technique that has claimed much attention
are   heuristics that decompose the total space into many smaller problems, and then which use a simpler optimizer for each region. 
For example, in $\mathcal{E}$-domination~\cite{deb05}, the  user is asked
`what is the lower threshold $\mathcal{E}$ on the size of a useful effect?''. The solution space
is then divided into boxes of size $\mathcal{E}$ and linked such that  the  set $X.\mathit{lower}$ contains boxes with worse objective scores that $X$.  Solutions in pairs boxes are  quickly compared  using   small samples from each  and, if some box $X$ is found to be inferior, then it is quickly pruned along with all
solutions in the $X.\mathit{lower}$ boxes.
Later research generalized this approach. MOEA/D (multiobjective
evolutionary algorithm based on decomposition~\cite{zhang07}) is a generic framework that decomposes a multiobjective optimization problem into many smaller single problems, then applies a second optimizer to each smaller subproblem, simultaneously.   Other work in this arena are the response surface methods
that quickly find multiple approximations to the problem, each of which holds for a very tiny region.
Each region has a ``slope'' and examples in that region are pushed along the slope towards better
solutions~\cite{krall15,Zuluaga:13}.
 
 
\input{algo} 
 \subsubsection{Selecting a Tuning Algorithm}
 
From all the above methods, how do we select which optimizers to apply to tuning data miners.
Cohen~\cite{cohen95} advises comparing any supposedly more
sophisticated method against the simplest possible alternative. For
example, in one study with ``floor effects'', Holte showed that,
often, much of the performance of complex multi-level decision trees
could be easily achieved using a much simpler single-level decision
tree learner called 1R~\cite{holte93}. He therefore recommends a very simple rule learner
(called ``1R'') as a
kind of ``scout'' that can do a quick preliminary analysis of a data
set and which can report back if that data really requires a more
complex analysis.

To find our ``scout'',  we used engineering judgement to sort  the above algorithms from simplest to most complex.
The three simplest optimizers are SA, $\mathcal{E}$-domination, and 
differential evolution (each can be coded in less than a page of some high-level scripting language). Our reading of the current literature is that there are more  advocates for
differential evolution than
  SA or $\mathcal{E}$-domination:
  \bi
  \item
  When the MOEA/D community requires a secondary optimizer, they often use  differential evolution~\cite{zhang07,5583335}.
  \item
 Vesterstrom and Thomsen~\cite{Vesterstrom04} report that DE is competitive with 
   particle swarm optimization and a genetic algorithm. 
   \ei
DEs have been applied before for   parameter tuning (e.g. see~\cite{omran2005differential, chiha2012tuning}) but this is the first time they have been applied to
optimizing defect prediction from static code attributes.  


 
 


 
  
 
The psuedocode for differential evolution is shown in Algorithm~\ref{alg:DE}.
Note that, as we describe the algorithm,
  any superscript number denotes a line in that algorithm.


DE is an evolutionary algorithm where next {\em NewGeneration} is learnt from
a current {\em Population}.  If the new is no better than  current, then
we lose one life (terminating when lives is zero)$^5$.

Each candidate solution in the {\em Population}  
is a pair of {\em (Tunings, Scores)}.  {\em Tunings} are selected from
\fig{parameters} and {\em Scores} come from training a learner using those parameters
and applying it     test data$^{23-28}$.

The premise of DE  is that the best way to mutate existing tunings
is to {\em Extrapolate}$^{29}$
between current solutions.  Three solutions $a,b,c$ are selected at random.
For each tuning parameter $i$, at some probability {\em cr}, we replace
the old tuning $x_i$ with $y_i$:
\bi
\item (For numerics) $y_i = a_i+f \times (b_i - c_i)$   where $f$ is a parameter
controlling the cross-over amount.  The {\em trim} function$^{39}$ limits the new
value to the legal range min..max of that parameter.
\item (For booleans) $y_i= \neg x_i$ (see line 37).
\ei
The main loop of DE$^7$ runs over the {\em Population}, replacing old items
with new {\em Candidate}s (if the new candidate is better than the old item).
This means that, as the loop progresses, the {\em Population} is full of increasiningly
more valuable solutions. This, in turn, also improves  the candidates, which are generated
from the {\em Population}.

For the experiments of this paper, we collect performance
values from a data mining, from which a {\em Goal} function extracts one 
performance value$^{27}$ (so we run this code many times, each time with
a different {\em Goal}$^2$).  Technically, this makes a  {\em single objective} DE (and for notes on multi-objective DEs, see~\cite{Coello05,zhang07,5583335}).


%\begin{algorithm}
%\begin{algorithmic}[1]
% \KwData{this text}
% \KwResult{how to write algorithm with \LaTeX2e }
% initialization\;
% \While{not at end of this document}{
%  read current\;
%  \eIf{understand}{
%   go to next section\;
%   current section becomes this one\;
%   }{
%   go back to the beginning of current section\;
%  }
% }
% \caption{How to write algorithms}
% \end{algorithmic}
%\end{algorithm}



\input{algo} 

\section{Experimental Design}\label{sect:design}

The following experiment aims to compare the performance of three learners, tuned and untuned, on 17
sets of data. 

\subsection{Data Sets}\label{sect:dataa}

Our defect data comes from the PROMISE repository\footnote{http://openscience.us/repo}.
It pertains to 
open source Java systems defined in terms of \fig{ck}:  {\it ant}, {\it camel}, {\it ivy}, {\it jedit}, {\it log4j}, {\it lucene},
{\it poi},{\it synapse}, {\it velocity} and {\it xerces}. 

An important principle in data mining is not to test on the data used
in training.  There are many ways to design a experiment that satisfies this principle.
Some of those methods have  limitations:
\bi
\item  Leave-one-out is too slow for large data sets;
\item Cross-validation can mix up older and newer data sets so it may well be that
data from the {\em future} is used to test on {\em past data}.
\ei
To avoid these problems, we used an incremental learning approach. The following
experiment ensures that the training data was created at some time before the test
data.

For this experiment, we use data sets with at least three  
consecutive releases  (where release $i+1$ was built after release $i$).
\bi 
\item The {\em first} release was used for some  {\em training}, to collect a baseline
   using an untuned learner. This release is also used  on line 24 of Algorithm~\ref{alg:DE} to
   build some model using some the tunings found in some {\em Candidate}.
   \item The {\em second} release was used on 25 of Algorithm~\ref{alg:DE} to 
   test the model found on line 24.
   \item Finally the {\em third} release was used to gather the performance statistics
   reported below from (a)~the model generated by the untuned learner or (b)~the
   best model found by differential evolution.
   \ei
This approach ensures   all treatments 
are assessed on the same tests. Note that we did consider
one other experimental design but rejected it for reasons of
internal validity (see \tion{construct}).

Some data sets have more than three releases and, for those data, we could run more
 than one experiment. For example, {\em ant} has five versions in PROMISE so
 we ran three experiments called V0,V1,V2:
 \bi
 \item AntV0: first,second,third = versions 1,2,3
 \item AntV1: first,second,third = versions 2,3,4
 \item AntV2: first,second,third = versions 3,4,5
 \ei 
These data sets are displayed in \fig{data1}.



 \begin{figure*}[!t]

\renewcommand{\baselinestretch}{0.8}
\scriptsize
\centering
  \begin{tabular}{c c c c c c c c c c }\hline
  Dataset &antV0&antV1&antV2&camelV0&camelV1&ivy&jeditV0&jeditV1&jeditV2
\\\hline
  training &20/125 &40/178 &32/293 &13/339 &216/608 &63/111 &90/272 &75/306 &79/312
\\  tuning  &40/178 &32/293 &92/351 &216/608 &145/872 &16/241 &75/306 &79/312 &48/367
\\  testing &32/293 &92/351 &166/745 &145/872 &188/965 &40/352 &79/312 &48/367 &11/492
\\ \hline
  Dataset &log4j&lucene&poiV0&poiV1&synapse&velocity&xercesV0&xercesV1
\\\hline
  training &34/135 &91/195 &141/237 &37/314 &16/157 &147/196 &77/162 &71/440
\\  tuning  &37/109 &144/247 &37/314 &248/385 &60/222 &142/214 &71/440 &69/453
\\  testing &189/205 &203/340 &248/385 &281/442 &86/256 &78/229 &69/453 &437/588
\\  \end{tabular}

   \caption{Data used in this data. 
   E.g., the top left data set has 20 defective classes out of 125 total.
   See \tion{dataa} for explanation of {\em training, tuning, testing} sets.
   }\label{fig:data1}
\end{figure*} 

\subsection{Optimization Goals}

Recall from Algorithm~1 that we call differential evolution once for each
 optimization goal. This section lists those optimization goals.

Let $\{A,B,C,D\}$ denote the
true negatives, 
false negatives, 
false positives, and 
true positives
(respectively) found by a binary detector. 
Certain standard measures can be computed from
$A,B,C,D$: 

{\scriptsize\[
\begin{array}{ll}
pd=recall=&D/(B+D)\\
pf=&C/(A+C)\\ 
prec=precision=&D/(D+C) \\
F =&2*pd*prec/(pd + prec)
\end{array}
\]}

All the above vary from zero to one. For $pf$, the {\em better} scores are {\em smaller}.
For all other scores, the {\em better} scores are {\em larger}.
One technical detail:   {\em prec}
and {\em F} refer to {\em both} the defect and non-defective modules. This is different to {\em pf} and {\em recall} which only
refer to either non-defective or defective modules (repsectively).


\begin{figure}[!b]
\small
\begin{tabular}{|p{.95\linewidth}|}\hline
One style
of measure is to check the {\em variability} of that predictor.
For example,
in their study on reproducibility of SE results,
 Anda, Sjoberg and Mockus advocate using the coefficient of variation ($CV=\frac{stddev}{mean}$).
Using this measure, they defined {\em reproducibility} as $\frac{1}{CV}$~\cite{anda09}.

There exist other goals that combine defect prediction with other economic
factors.
Arisholm~\&~Briand~\cite{arisholm06},  Ostrand \& Weyeuker~\cite{ostrand04} and Rahman et al.~\cite{rahman12}
say that a defect predictor should maximizing {\em reward}; i.e. find the fewest lines of code
that contain the most bugs.

Yin et al. are concerned about
 {\em incorrect bug fixes}; i.e. those that require subsequent work in order to complete the bug fix.
These bugs occur  when (say) developers try to fix parts of the code
where they have very little experience~\cite{yin11}.  To avoid such incorrect bug fixes, we have to optimize
for finding the most number of bugs in regions that {\em the most programmers have worked with before}.

In {\em Better-faster-cheaper},   managers want changes that lead
to fewer defects,  faster development,  using less resources~\cite{Green,elrawas08,elrawas10,me07f,me09a,me09f}.

Another  {\em  rush-to-market} approach is yet another economic-based optimization measure.
A learner that tries to maximize ``rush-to-market'' is trying to release the product as soon
as possible, without too many bugs. Note that ``rush-to-market'' is an appropriate strategy for a company competing
in a volatile and crowded market place where being first-to-market enables a revenue stream (that can be
used to subsequently fix any issues with version 1.0)~\cite{huang06}.\\\hline
\end{tabular}
\caption{Many ways to assess defect predictors.}\label{fig:criteria}
\end{figure}


Our analysis results make no assumption that (e.g.) minimizing false alarms is 
more important than maximizing   precision. That determination 
depends on current business conditions:
\be
\item
For safety critical applications, high false alarm rates may be acceptable if the cost
of overlooking any critical can outweigh the inconvenience of having to inspect a few more
modules. 
\item
On the other hand, when rushing a product to market before a competing product is released, there is a business case to 
avoid the extra rework associated with false alarms.  In that business context, 
managers might be willing to lower the recall somewhat in order to minimize the false alarms.
\ee
These two examples are just the tip of the iceberg (see other criteria in \fig{criteria})
and there is insufficient space in this paper to explore all the above optimization goals.
What we can do, however, is show examples of how  changing  optimization goals can also change 
the conclusions made from that learner on that data. Those examples only relate to precision, recall, and the F-measure
but the general principle (that the search bias changes the search conclusions) would hold for all the goals
listed in the previous paragraph.  Hence, we warn that it is important not to overstate  empirical results from software analytics.
Rather, those results need to be expressed {\em along with} the context within which they are
relevant (and by ``context'', we mean the optimization goal).


\section{Experimental Results}

The introduction of this article made several claims about tuning defect predictor
that use static code attributes. This section presents
support for those claims:
\be
\item  Tuning    improves the performance scores of a predictor;
e.g.  precision changes from 2\% to 98\%: see \tion{precision};
\item Tuning changes conclusions on what learners are better than others: see \tion{rank};
\item Tuning changes conclusions about what factors are most important: see \tion{import};
\item  Tuning is easy: see \tion{easy};
\item Tuning is fast: see \tion{fast};
\item Data miners should not be used ``off-the-shelf'' with their default tunings: see \tion{variance}.
\ee


One comment before continuing. The following text does not
show results for tuning on recall
or false alarms. The reason is that false alarm and recall focus on {\em only}
the non-defective and defective modules (respectively). Hence:
\bi
\item
When DE tunes for recall, we can achieve near
100\% recall-- but the cost of a near 100\% false alarms.
\item
Similarly, the tuner
could achieve near zero false alarm rates by effectively turning off
the detector (driving recall to zero).
\ei
Hence, in the
following, we explore performance measures that comment on all the possible
target classes (i.e. precision and ``F''). 


\begin{figure}[!b]
\renewcommand{\baselinestretch}{0.8} 

\scriptsize    

\begin{tabular}{r|rl|rl|rl|rl|rl|rlrl}
%\begin{tabular}{r@{~}|r@{~}l@{~}|r@{~}l@{~}|r@{~}l|r@{~}l@{~}|r@{~}l@{~}|r@{~}l@{~}r@{~}l}
      &   \multicolumn{4}{c|}{WHERE}         &   \multicolumn{4}{c|}{CART}         &   \multicolumn{4}{c}{Random Forest}         \\\hline
  Data set   &   \multicolumn{2}{c}{default}         &   \multicolumn{2}{c|}{Tuned}         &   \multicolumn{2}{c}{default}         &   \multicolumn{2}{c|}{Tuned}    &   \multicolumn{2}{c}{default}  &   \multicolumn{2}{c}{Tuned}\\\hline
antV0 & 30 &   & {\bf 89} &   & 27 &   & {\bf 89} &   & 28 &   & {\bf 89} &  \\
antV1 & 32 &   & {\bf 74} &   & 36 &   & {\bf 74} &   & 41 &   & {\bf 74} &  \\
antV2 & 78 &   & 52 &   & 52 &   & 56 &   & 64 &   & {\bf 100} &  \\
camelV0 & {\bf 83} &   & {\bf 83} &   & 26 &   & 34 &   & 50 &   & 78 &  \\
camelV1 & 22 &   & 81 &   & 24 &   & {\bf 83} &   & 25 &   & 31 &  \\
ivy & 16 &   & {\bf 89} &   & 17 &   & 25 &   & 16 &   & 21 &  \\
jeditV0 & 35 &   & 48 &   & 49 &   & {\bf 61} &   & 44 &   & 50 &  \\
jeditV1 & 24 &   & {\bf 87} &   & 28 &   & 62 &   & 28 &   & 37 &  \\
jeditV2 & 2 &   & {\bf 98} &   & 3 &   & 18 &   & 5 &   & 5 &  \\
log4j & 94 &   & {\bf 100} &   & 97 &   & {\bf 100} &   & 99 &   & {\bf 100} &  \\
lucene & 61 &   & 76 &   & 67 &   & 70 &   & 67 &   & {\bf 77} &  \\
poiV0 & 74 &   & 74 &   & {\bf 79} &   & 75 &   & 77 &   & 78 &  \\
poiV1 & {\bf 100} &   & 75 &   & 73 &   & 89 &   & 82 &   & {\bf 100} &  \\
synapse & 66 &   & 66 &   & 71 &   & {\bf 100} &   & 65 &   & {\bf 100} &  \\
velocity & 34 &   & {\bf 43} &   & 34 &   & 39 &   & 34 &   & 42 &  \\
xercesV0 & 13 &   & {\bf 85} &   & 14 &   & 47 &   & 15 &   & 14 &  \\
xercesV1 & {\bf 56} &   & 26 &   & 49 &   & 26 &   & 49 &   & 26 &  \\
\end{tabular}
\caption{Precision results (best results  shown in {\bf bold}).}
\label{fig:precisionbars}
\end{figure}

\subsection{Tuning Can  Improves Performance Scores}\label{sect:precision}

\fig{precisionbars} shows precision results before and after tuning using three different learners.
For each data set, the maximum precision values for each data set are shown in {\bf bold}.


These results offer support for the prior conclusions of  
 Lessmann et al.~\cite{lessmann2008benchmarking} (that CART is worse than Random Forest):
\bi
\item
Untuned CART is indeed the worst learner (only one of its
untuned results is best and {\bf bold}). 
\item 
In $\frac{13}{17}$ cases, the performance of untuned Random Forest is better than or equal to untuned CART.  
\ei
% Some of the data sets in \fig{precisionbars} proved challenging for all learners;
% e.g. the precision results for {\em ivy} are less
% that impressive.
% To some extent, this is due to
% the properties of   the data set (as shown in \fig{data1}, defective classes in {\em ivy} are very rare in tuning data).

That said,  tuning can improve those poor performing detectors:
\bi
\item
For {\em jeditV2}, nearly all learners report precisions of under 20\%. But tuned  WHERE scores a 98\% precision.  
\item A similar pattern can be seen in results from {\em ivy},   and {\em xercesV0}.
In those two data sets, nearly all the precision values are   low {\em except} for
tuned WHERE that scored 89\% and 85\%.
\ei
Finally, note the  {\em jeditV2} result for the WHERE learner.
Here, tuning changes precision from 2\% to 98\%.

\fig{deltas} shows that tuning usually has a positive effect on performance scores. This figure sorts
the deltas in the precision and the F-measure    between tuned and untuned learners. Our reading of this
figure is that, overall, tuning rarely makes performance   worse and usually can make it much better. 
 



\begin{figure}[!t]
\begin{center}
\includegraphics[width=1.25in]{precision_W.eps}\includegraphics[width=1.25in]{F_W.eps}
 \end{center}
\caption{Deltas in performance  seen in \fig{precisionbars} (left)
and \fig{fbars} (right) between tuned and untuned learners. Tuning improves performance when the deltas are above zero.}\label{fig:deltas}
 \end{figure}

\subsection{Tuning Changes the Rankings of Learner  }\label{sect:rank}
Researchers often use performance criteria to assert that one learner is better than another~\cite{lessmann2008benchmarking,hall11,me07b}.
Such conclusions are   not reliable since they can change when we elect to tune for different goals. 

For example, given the poor showing of untuned CART and the good performance of tuned WHERE
in   \fig{precisionbars}, we might recommend
{\em not} to use CART and {\em always} used tuned WHERE. In turns out that this
conclusion is not stable across the tuning process.


\fig{fbars} shows what happens when we tune for the f-measure. 
As before, untuned Random Forest generally beats CART. However:
\bi
\item
In  a result that is the direct opposite of   
 Lessmann et al.~\cite{lessmann2008benchmarking}, tuned CART does better than Random Forest;
 \item 
 In another result that is the reverse of \fig{precisionbars}, tuned WHERE is not necessarily
 any better than anything else.
 \ei
The lesson here is that, given a particular task (e.g. optimizing for precision), tuning can offer
substantial benefits. However, when that task changes (e.g. to optimizing for the F-measure),
it should not be assumed that conclusions from the previous tuning will hold in the new context.
In practice, this means that tuning needs to be repeated for each new context.





\begin{figure}[!t]
\renewcommand{\baselinestretch}{0.8} 

\scriptsize  
~~~\begin{tabular}{r|rl|rl|rl|rl|rl|rlrl}
      &   \multicolumn{4}{c|}{WHERE}         &   \multicolumn{4}{c|}{CART}         &   \multicolumn{4}{c}{Random Forest}         \\\hline
  Data set   &   \multicolumn{2}{c}{default}         &   \multicolumn{2}{c|}{Tuned}         &   \multicolumn{2}{c}{default}         &   \multicolumn{2}{c|}{Tuned}    &   \multicolumn{2}{c}{default}  &   \multicolumn{2}{c}{Tuned}\\\hline
antV0 & 39 &   & 20 &   & 32 &   & 40 &   & {\bf 41} &   & 21 &  \\
antV1 & 11 &   & 5 &   & 38 &   & 49 &   & 37 &   & {\bf 55} &  \\
antV2 & 0 &   & 3 &   & 44 &   & {\bf 48} &   & 47 &   & 46 &  \\
camelV0 & 0 &   & {\bf 91} &   & 9 &   & 28 &   & 4 &   & 31 &  \\
camelV1 & 35 &   & 35 &   & 31 &   & 33 &   & {\bf 37} &   & 33 &  \\
ivy & 28 &   & 32 &   & 28 &   & 30 &   & 27 &   & {\bf 33} &  \\
jeditV0 & 50 &   & {\bf 57} &   & 56 &   & {\bf 57} &   & 56 &   & {\bf 57} &  \\
jeditV1 & 37 &   & 38 &   & 36 &   & 45 &   & 43 &   & {\bf 49} &  \\
jeditV2 & 4 &   & 6 &   & 5 &   & {\bf 9} &   & {\bf 9} &   & 8 &  \\
log4j & {\bf 61} &   & 54 &   & 47 &   & 46 &   & 57 &   & 45 &  \\
lucene & 70 &   & 74 &   & 56 &   & 74 &   & 70 &   & {\bf 75} &  \\
poiV0 & 82 &   & 71 &   & 79 &   & 62 &   & {\bf 83} &   & 73 &  \\
poiV1 & 5 &   & {\bf 78} &   & 21 &   & {\bf 78} &   & 19 &   & {\bf 78} &  \\
synapse & 0 &   & 2 &   & 40 &   & 55 &   & 41 &   & {\bf 56} &  \\
velocity & 51 &   & 51 &   & 49 &   & {\bf 53} &   & 51 &   & 51 &  \\
xercesV0 & 22 &   & 23 &   & 21 &   & {\bf 27} &   & 24 &   & 20 &  \\
xercesV1 & 23 &   & 4 &   & 18 &   & {\bf 47} &   & 18 &   & 40 &  \\
\end{tabular}
\caption{F-value results (best results  shown in {\bf bold}).}
\label{fig:fbars}
\end{figure}


\subsection{Tuning Changes the Ranking of Project Factors  }\label{sect:import}


Researchers often use data miners to  test what factors have most impact on software projects~\cite{bell2013limited,rahman2013how,me02k,moser2008comparative,zimmermann2007predicting,herzig2013predicting}. 
\fig{counts} shows that such conclusions are not reliable since they can be changed by tuning.
This figure shows what attributes appeared in WHERE's final decision trees when optimizing for different goals.
Note that the counts in the {\em tuned} column are   usually
different to those in the {\em default} columns (the tuned learners used more attributes
than otherwise as witnessed by the taller columns of numbers in the ``tuned'' column). 




\subsection{Tuning is Easy}\label{sect:easy}

In terms of the search space
explored via tuning, optimizing defect prediction from static code
measures is much {\em smaller} than standard optimization.

To see this,
recall from Algorithm~1 that
DE explores a {\em Population} of size {\em np=10}. This is a very small population size since
Rainer Storn (one of the inventors of DE) recommends  setting {\em np} to be ten times larger than the number
of attributes being optimized~\cite{storn1997differential}.

From \fig{parameters},
we see that Storn would therefore recommend {\em np} values of
90, 50, 60 for WHERE, CART and Random Forests (respectively). Yet we achieve our results
using a constant {\em np=10}; i.e. $\frac{10}{90}, \frac{10}{50}, \frac{10}{60}$ of the
recommended search space.

Another measure showing that tuning is easy 
(for static code defect predictors)
is the number of evaluations required to complete optimization
(see next section).



\begin{figure}[!ht]

\renewcommand{\baselinestretch}{0.8}
\scriptsize
\centering
  \begin{tabular}{c|c c|c c|c c|c c| c c }
  
    &   \multicolumn{2}{c|}{Precision} & \multicolumn{2}{c|}{F} &  \multicolumn{2}{c|}{SUM}\\
 &&&&&&&\\
Features&   
  default
& tuned
& default
& tuned
& default
& tuned
\\\hline

max\_cc& & & &  & & \\
noc& & & & & & \\
ca& & & & & & \\
cbo& & 1& & & & 1\\
moa& & 1& & & & 1\\
ce& & 2& & 1& & 3\\
avg\_cc& & 2& & 2& & 4\\
npm& 1& 2& 1& 1&2 & 3\\
lcom& 1& 2& 1& 1& 2& 3\\
amc& 4& 2& 4& 2& 8& 4\\
cbm& 5& 2& 5& 3& 10& 5\\
rfc& 3& 6& 3& 8& 6& 14\\
wmc& 5& 4& 5& 9& 10& 13\\
dit& 8& 3& 7& 7& 15& 10\\
ic& 8& 3& 8& 6& 16 & 9\\
lcom3& 8& 6& 8& 8&16 & 14\\
cam& 9& 7& 9& 7& 18& 14\\
loc& 9& 5& 9& 11& 18& 16\\
dam& 13& 6& 13& 11& 26& 17 \\
mfa& 16& 9& 16& 9&32 & 18\\

 
  \end{tabular}
    \caption{Counts of features selected by different goals.Given that we are processing 17 data sets, the maximum counts for any 
one cell in the ``precision'' or ``F'' column is 17.  
    }\label{fig:counts}
\end{figure}
 

\begin{figure*}[!ht]

\renewcommand{\baselinestretch}{0.8}
\scriptsize
\centering
  \begin{tabular}{r|c |c |c |c |c |c }
    \hline\hline
    Datasets & Tuned\_Where & Default\_Where & Tuned\_CART & Default\_CART & Tuned\_RanFst & Default\_RanFst\\
    \hline
    antV0 & 50 / 90.18 & 1.57 & 60 / 4.36 & 0.09 & 50 / 8.12 & 0.18\\
    antV1 & 50 / 174.67 & 2.90 & 50 / 6.35 & 0.10 & 50 / 11.77 & 0.27\\
    antV2 & 50 / 403.63 & 6.92 & 70 / 9.71 & 0.16 & 60 / 13.28 & 0.35\\
    camelV0 & 50 / 537.53 & 8.60 & 50 / 9.14 & 0.18 & 50 / 13.73 & 0.31\\
    camelV1 & 60 / 1640.54 & 24.57 & 60 / 17.06 & 0.24 & 70 / 31.53 & 0.73\\
    ivy & 70 / 77.75 & 1.02 & 60 / 3.86 & 0.06 & 50 / 8.00 & 0.17\\
    jeditV0 & 80 / 472.57 & 5.49 & 60 / 6.30 & 0.09 & 70 / 13.01 & 0.30\\
    jeditV1 & 60 / 489.45 & 6.82 & 70 / 7.61 & 0.11 & 60 / 12.87 & 0.32\\
    jeditV2 & 50 / 435.43 & 7.21 & 50 / 6.46 & 0.12 & 90 / 20.34 & 0.37\\
    log4j & 70 / 113.73 & 1.36 & 70 / 3.25 & 0.05 & 60 / 7.07 & 0.16\\
    lucene & 70 / 224.39 & 2.70 & 50 / 4.07 & 0.08 & 50 / 8.87 & 0.26\\
    poiV0 & 60 / 261.06 & 4.00 & 60 / 6.23 & 0.10 & 50 / 10.57 & 0.29\\
    poiV1 & 80 / 607.85 & 7.18 & 60 / 7.69 & 0.13 & 50 / 11.39 & 0.29\\
    synapse & 50 / 116.04 & 1.87 & 60 / 4.07 & 0.05 & 70 / 9.74 & 0.16\\
    velocity & 60 / 195.27 & 2.75 & 60 / 4.49 & 0.06 & 80 / 12.15 & 0.21\\
    xercesV0 & 60 / 143.69 & 2.17 & 70 / 7.26 & 0.09 & 60 / 10.28 & 0.23\\
    xercesV1 & 50 / 794.50 & 13.37 & 50 / 8.24 & 0.15 & 60 / 14.54 & 0.38\\

  \end{tabular}
  \caption{Evaluations/runtimes  (in seconds), optimizing for  precision.  }\label{fig:etimes}
\end{figure*}


%%%%time for F %%%%%%
\begin{figure*}[!ht]
\renewcommand{\baselinestretch}{0.8}
\scriptsize
\centering
  \begin{tabular}{r|c |c |c |c |c |c }
    \hline\hline
    Datasets & Tuned\_Where & Default\_Where & Tuned\_CART & Default\_CART & Tuned\_RanFst & Default\_RanFst\\
    \hline
    antV0 & 50 / 94.08 & 1.71 & 60 / 4.55 & 0.08 & 60 / 10.79 & 0.21\\
    antV1 & 60 / 193.74 & 3.02 & 70 / 7.77 & 0.09 & 60 / 12.30 & 0.25\\
    antV2 & 80 / 643.94 & 7.59 & 60 / 8.38 & 0.15 & 70 / 16.99 & 0.41\\
    camelV0 & 60 / 662.56 & 9.97 & 60 / 13.19 & 0.23 & 80 / 26.11 & 0.32\\
    camelV1 & 60 / 1800.64 & 24.25 & 50 / 15.02 & 0.28 & 50 / 28.52 & 0.78\\
    ivy & 60 / 69.95 & 1.03 & 50 / 3.35 & 0.08 & 70 / 9.40 & 0.18\\
    jeditV0 & 90 / 553.80 & 5.58 & 50 / 5.58 & 0.09 & 60 / 15.08 & 0.33\\
    jeditV1 & 60 / 519.75 & 8.76 & 50 / 7.43 & 0.13 & 60 / 18.13 & 0.41\\
    jeditV2 & 70 / 621.32 & 8.98 & 50 / 9.71 & 0.15 & 60 / 17.38 & 0.63\\
    log4j & 70 / 125.29 & 1.73 & 50 / 2.90 & 0.06 & 60 / 8.76 & 0.19\\
    lucene & 50 / 221.99 & 3.52 & 50 / 5.20 & 0.10 & 50 / 10.09 & 0.33\\
    poiV0 & 60 / 327.48 & 5.13 & 50 / 6.56 & 0.11 & 50 / 12.88 & 0.36\\
    poiV1 & 50 / 523.85 & 8.95 & 80 / 12.26 & 0.14 & 60 / 19.56 & 0.35\\
    synapse & 70 / 148.23 & 1.91 & 60 / 3.96 & 0.06 & 60 / 8.19 & 0.16\\
    velocity & 50 / 156.51 & 2.75 & 60 / 4.27 & 0.06 & 50 / 7.70 & 0.22\\
    xercesV0 & 60 / 142.83 & 2.01 & 70 / 7.15 & 0.08 & 60 / 9.61 & 0.20\\
    xercesV1 & 50 / 751.92 & 12.98 & 60 / 9.28 & 0.16 & 50 / 12.69 & 0.38\\

  \end{tabular}
  \caption{Evaluations/runtimes  (in seconds), optimizing for  ``F''.  }\label{fig:ftimes}
\end{figure*}



\subsection{Tuning is Fast}\label{sect:fast}
 
 
The number of evaluations/runtimes used by our optimizers is shown in \fig{etimes}
and \fig{ftimes}.
WHERE's runtimes are slower than CART and Random Forest since WHERE has yet to benefit from decades
of implementation experience with these older algorithms. For example, SciKitLearn's  CART and Random Forest
 make extensive use of an underlying ``C'' library whereas WHERE is a purely interpreted Python.

Looking over \fig{etimes} and \fig{ftimes}, the general pattern is that 50 to 80 evaluations suffice for finding the tuning
improvements reported in this paper. 
50 to 80 evaluations is  much less than our pre-experimental intuition.
Prior to this paper, the authors have conducted numerous explorations of evolutionary algorithms
for search-based SE applications~\cite{krall15,krall15:hm,fea02a,me07f,Green}. Based
on that work, our expectations were that non-parametric evolutionary optimization would
take thousands, if not millions, of evaluations of candidate tunings. Hence, originally,
we used Storn's advise on the size of {\em np} and ran our optimizers for 24 hours. 
Then, just
to get a baseline, we tried {\em np=10} and the early stopping rules (number of lives used in
Algorithm~1 on line 5) and obtained the results of this paper



\subsection{Do not use ``off-the-shelf''  Default Tunings}\label{sect:variance}
 
 \fig{preselect} and \fig{fselect} show   tunings learned by DE. 
The left hand side columns shows some standard default values. Note that:
\bi
\item
The tunings learned by DE
were often very different to the default. That is, to achieve the performance improvements seen in the paper,
the default tuning parameters required a wide range of adjustments.
\item The tunings learned by DE were different in different data sets and for different goals.
That is, to achieve the improvements seen in this paper, tuning has to be repeated whenever the goals or data
sets are changed.
\ei
Given this requirement to repeatedly run tuning, it is fortunate that (as shown above)
tuning is so easy and so fast (at least for defect predictors from static code attributes).




 
%%%%parameters for prec %%%%%%
\begin{figure*}[!ht]
 
\renewcommand{\baselinestretch}{0.9}
\resizebox{\textwidth}{!}{
\scriptsize
\centering
  \begin{tabular}{|c |c |c |c |c |c |c |c |c |c |c |c |c |c |c |c |c |c |c |c |}
    \hline
  \begin{tabular}[c]{@{}c@{}}Learner \\ Name\end{tabular}&Parameters  & Default &antV0&antV1&antV2&camelV0&camelV1&ivy&jeditV0&jeditV1&jeditV2&log4j&lucene&poiV0&poiV1&synapse&velocity&xercesV0&xercesV1\\ 
 \hline
\multirow{8}{*}{\begin{tabular}[c]{@{}c@{}}Where\\based\\ Learner\end{tabular}}
& threshold& 0.5& 0.98& 0.98& 0.43& 0.24& 0.64& 1& 1& 0.98& 0.98& 1& 1& 0.87& 0.59& 0.98& 1& 0.98& 0.98\\ \cline{2-20}
& infoPrune& 0.33& 0.05& 0.05& 0.71& 0.54& 0.45& 0.41& 0.3& 0.05& 0.05& 0.54& 0.84& 0.01& 1& 0.05& 0.68& 0.43& 0.05\\ \cline{2-20}
& min\_sample\_size& 4& 7& 7& 9& 8& 6& 10& 1& 5& 7& 8& 7& 9& 3& 7& 7& 1& 7\\ \cline{2-20}
& min\_Size& 0.5& 0.51& 0.51& 0.59& 0.46& 0.13& 0.38& 0.66& 0.27& 0.51& 0.46& 0.47& 0.77& 0.48& 0.51& 0.66& 0.22& 0.51\\ \cline{2-20}
& wriggle& 0.2& 0.6& 0.6& 0.83& 0.52& 0.19& 0.01& 0.26& 0.6& 0.6& 0.52& 0.19& 0.83& 0.01& 0.6& 0.26& 0.55& 0.6\\ \cline{2-20}
& depthMin& 2& 1& 1& 2& 3& 5& 2& 3& 3& 1& 1& 2& 4& 2& 1& 3& 2& 1\\ \cline{2-20}
& depthMax& 10& 8& 8& 13& 19& 19& 18& 7& 8& 8& 19& 1& 19& 18& 8& 11& 18& 8\\ \cline{2-20}
& wherePrune& False& False& False& False& False& True& True& False& False& False& False& False& True& True& False& False& True& False\\ \cline{2-20}
& treePrune& True& False& False& False& True& True& True& False& False& False& True& True& False& False& False& False& True& False\\ \cline{2-20}
\hline
\multirow{4}{*}{CART}
& threshold& 0.5& 0.69& 0.99& 1& 0.3& 0.83& 1& 0.99& 0.58& 0.72& 1& 0.71& 0.46& 0.72& 1& 0.85& 1& 0.64\\ \cline{2-20}
& max\_feature& None& 0.01& 0.58& 0.65& 0.66& 0.73& 0.67& 0.56& 0.01& 0.97& 0.54& 0.52& 0.32& 0.01& 0.74& 0.73& 0.01& 0.1\\ \cline{2-20}
& min\_samples\_split& 2& 7& 16& 18& 5& 11& 6& 15& 6& 17& 4& 16& 12& 5& 14& 11& 4& 10\\ \cline{2-20}
& min\_samples\_leaf& 1& 13& 14& 10& 4& 3& 15& 16& 9& 6& 7& 6& 1& 4& 6& 7& 11& 7\\ \cline{2-20}
& max\_depth& None& 14& 1& 41& 34& 1& 22& 29& 1& 1& 1& 14& 19& 8& 40& 4& 1& 1\\ \cline{2-20}
\hline
\multirow{6}{*}{\begin{tabular}[c]{@{}c@{}}Random \\ Forests\end{tabular}} 
& threshold& 0.5& 0.84& 0.9& 0.83& 0.33& 1& 0.99& 0.91& 1& 1& 0.83& 0.98& 0.9& 0.86& 0.83& 1& 1& 1\\ \cline{2-20}
& max\_feature& None& 0.61& 0.13& 0.89& 0.37& 0.01& 0.98& 0.52& 0.75& 0.35& 0.01& 0.98& 0.84& 0.73& 0.01& 0.48& 0.51& 0.01\\ \cline{2-20}
& max\_leaf\_nodes& None& 37& 35& 38& 21& 36& 45& 10& 38& 10& 30& 20& 43& 11& 13& 15& 39& 10\\ \cline{2-20}
& min\_samples\_split& 2& 8& 16& 17& 13& 14& 2& 3& 2& 2& 18& 19& 9& 4& 4& 9& 20& 1\\ \cline{2-20}
& min\_samples\_leaf& 1& 19& 5& 2& 4& 2& 4& 7& 17& 7& 16& 12& 2& 3& 2& 2& 2& 3\\ \cline{2-20}
& n\_estimators& 100& 138& 112& 77& 74& 125& 130& 107& 85& 96& 111& 103& 82& 59& 149& 150& 63& 58\\ \cline{2-20}
\hline  \end{tabular}
}
  \caption{Parameters tuned on different models over the objective of precision.}\label{fig:preselect}
\end{figure*}


%%%%parameters for F %%%%%%
\begin{figure*}[!ht]
 
\resizebox{\textwidth}{!}{
\renewcommand{\baselinestretch}{0.9}
\scriptsize
\centering
  \begin{tabular}{|c |c |c |c |c |c |c |c |c |c |c |c |c |c |c |c |c |c |c |c |}
    \hline
    
  \begin{tabular}[c]{@{}c@{}}Learner \\ Name\end{tabular}&Parameters  & Default &antV0&antV1&antV2&camelV0&camelV1&ivy&jeditV0&jeditV1&jeditV2&log4j&lucene&poiV0&poiV1&synapse&velocity&xercesV0&xercesV1\\ 
 \hline
\multirow{8}{*}{\begin{tabular}[c]{@{}c@{}}Where\\based\\ Learner\end{tabular}}
& threshold& 0.5& 0.04& 0.44& 0.44& 0.98& 0.65& 0.77& 1& 0.65& 0.98& 0.44& 0.44& 0.87& 0.04& 0.77& 0.24& 0.44& 0.77\\ \cline{2-20}
& infoPrune& 0.33& 0.51& 0.68& 0.88& 0.47& 0.07& 0.31& 0.48& 0.68& 0.57& 0.12& 0.68& 0.01& 0.51& 0.14& 0.54& 0.68& 0.14\\ \cline{2-20}
& min\_sample\_size& 4& 6& 4& 6& 1& 6& 8& 8& 4& 6& 7& 4& 9& 6& 2& 8& 4& 8\\ \cline{2-20}
& min\_Size& 0.5& 0.18& 0.4& 0.56& 0.51& 0.65& 0.59& 0.97& 0.4& 0.51& 0.8& 0.4& 0.77& 0.18& 0.62& 0.46& 0.4& 0.66\\ \cline{2-20}
& wriggle& 0.2& 0.25& 0.29& 0.76& 0.6& 0.63& 0.26& 1& 0.51& 0.17& 0.36& 0.51& 0.83& 0.25& 0.5& 0.52& 0.29& 0.26\\ \cline{2-20}
& depthMin& 2& 3& 3& 3& 1& 5& 3& 2& 3& 5& 5& 3& 4& 3& 3& 3& 3& 3\\ \cline{2-20}
& depthMax& 10& 16& 15& 15& 8& 19& 10& 7& 15& 5& 15& 15& 19& 16& 6& 19& 15& 10\\ \cline{2-20}
& wherePrune& False& False& True& True& True& True& True& True& False& False& True& True& True& False& True& False& False& True\\ \cline{2-20}
& treePrune& True& False& True& True& False& False& False& False& False& True& True& True& False& False& False& True& True& False\\ \cline{2-20}
\hline
\multirow{5}{*}{CART}
& threshold& 0.5& 0.34& 0.25& 0.01& 0.01& 0.73& 0.53& 0.92& 0.8& 0.74& 0.54& 0.03& 0.91& 0.01& 0.01& 0.55& 1& 0.01\\ \cline{2-20}
& max\_feature& None& 0.01& 0.01& 0.29& 0.01& 0.46& 0.75& 0.79& 0.74& 0.41& 0.81& 0.61& 0.72& 0.01& 0.01& 0.01& 0.25& 0.18\\ \cline{2-20}
& min\_samples\_split& 2& 18& 20& 12& 2& 15& 11& 2& 18& 13& 9& 17& 16& 10& 4& 8& 3& 15\\ \cline{2-20}
& min\_samples\_leaf& 1& 19& 16& 15& 17& 1& 1& 13& 10& 4& 3& 7& 5& 20& 7& 8& 1& 6\\ \cline{2-20}
& max\_depth& None& 12& 2& 15& 1& 41& 20& 44& 15& 13& 5& 23& 14& 1& 5& 17& 47& 13\\ \cline{2-20}
\hline
\multirow{6}{*}{\begin{tabular}[c]{@{}c@{}}Random \\ Forests\end{tabular}} 
& threshold& 0.5& 0.01& 0.35& 0.3& 0.01& 0.9& 0.97& 0.63& 1& 0.73& 0.68& 0.01& 1.0& 0.01& 0.07& 0.22& 1& 0.82\\ \cline{2-20}
& max\_feature& None& 0.63& 0.17& 0.01& 0.01& 0.88& 0.74& 0.76& 0.73& 0.01& 0.03& 0.39& 0.02& 0.01& 0.56& 0.36& 0.51& 0.89\\ \cline{2-20}
& max\_leaf\_nodes& None& 40& 33& 46& 22& 11& 16& 38& 34& 30& 31& 12& 49& 25& 47& 15& 39& 24\\ \cline{2-20}
& min\_samples\_split& 2& 10& 16& 20& 1& 1& 1& 1& 4& 20& 19& 11& 14& 2& 17& 19& 20& 19\\ \cline{2-20}
& min\_samples\_leaf& 1& 4& 15& 9& 13& 18& 11& 3& 16& 17& 6& 10& 7& 19& 13& 11& 2& 14\\ \cline{2-20}
& n\_estimators& 100& 120& 73& 75& 130& 97& 144& 125& 97& 80& 111& 96& 101& 50& 67& 74& 63& 66\\ \cline{2-20}
\hline  \end{tabular}
}
  \caption{Parameters tuned on different models over the objective of ``F''.}\label{fig:fselect}
\end{figure*}


\section{Reliability and Validity}\label{sect:construct}


{\em Reliability} refers to the consistency of the results obtained
from the research.  For example,   how well independent researchers
could reproduce the study? To increase external
reliability, this paper has taken care to either  clearly define our
algorithms or use implementations from the public domain
(SciKitLearn). Also, all the data used in this work is available
on-line in the PROMISE code repository and all our algorithms
are on-line at github.com/a-se/where.


{\em Validity} refers to the extent to which a piece of research actually
investigates what the researcher purports to investigate.
{\em Internal validity} checks that if the differences found in
the treatments can be ascribed to the treatments under study. 
One internal validity issue with our experiments is the choice
of {\em training, tuning, and testing} data sets discussed in 
\tion{design}. Recall that while all our learners used the same
{\em testing} data set, our untuned learners were only given
access to {\em training} while DEs could get feedback from
{\em training} and
{\em tuning}.  This mean that, potentially,  the performance increases
reported above are merely an artificat of the DEs being able to access
more data. However:
\bi
\item Note that
DEs assess their performance using a model built from {\em training}
then applied to the {\em tuning} data. If we gave the untuned
learners double the training data, then this would be inherently
biased against DEs.
\item In practice, this is much ado about nothing. We have
repeated all the experiments of this paper with the untuned
learners building models from {\em training} + {\em tuning}.
While the exact numbers found in this way
may differ from the above in some small way, we can report that 
those repeated runs do not substantively change the above conclusions.
\ei
{\em External validity} checks if the results are of relevance
for other cases, or can be generalized from samples
to populations.  
The tuning results shown here only came from one  software analytics task 
(defect prediction from static code attributes).
There are many other kinds of software analytics tasks 
(software development effort estimation, social network mining,
detecting duplicate issue reports, etc) and the implication of this
study for those tasks is unclear. 
However,  those other tasks often use the same kinds of learners
explored in this paper so it is quite possible that
the conclusions of this paper apply to other SE analytics tasks as well. 

That said, there exist some class of data mining papers for which
tuning may not be required. Consider  Le Goues et al.'s 2012
ICSE paper that used a evolutionary program to learn
repairs to code~\cite{leGoues12}. The performance criteria
in that paper was ``can we fix any of the known bugs?''. Note
that this criteria is a ``{\em competency}'' statement, and
not a ``{\em better than}'' statement (the difference being that
one is 
``can do'' and the other is ``can do better''). For such
competency claims, tuning is not necessary. However, as soon
as {\em better than} enters the performance criteria then this
becomes a race between competing methods. In such a race,
it is unfair to hobble one competitor with poor tunings.



\section{Conclusions}

When learning defect predictors for static code
attributes,   analytics without parameter tuning is considered {\em harmful} and {\em misleading}:
\bi
\item Tuning improves the performance scores of a predictor.
That improvement is usually positive (see \fig{deltas}) and sometimes
it can be quite   dramatic (e.g. precision changing from 2\% to 98\%). \item 
Tuning changes conclusions on what learners are better than others.
Hence, it is time to revisit numerous prior publications of our own~\cite{me07b}
and others~\cite{lessmann2008benchmarking,hall11}.
\item
Tuning changes also changes conclusions on what factors are most important in software development.
Once again, this means that old papers may need to be revised including those
some of our own~\cite{me02k} and others~\cite{bell2013limited,rahman2013how,moser2008comparative,zimmermann2007predicting,herzig2013predicting}. 
\ei
Accordingly, we strongly advise that data miners should not be used ``off-the-shelf'' with their default tunings. 
\fig{preselect} and \fig{fselect}    show just how much tuning can alter default settings
both for different data sets and for different goals. 

Since the results from tuning are unstable,
tuning needs to be repeated
whenever data or goals are changed.
Fortunately, the cost of find good tunings is not excessive since, at least for
static code defect predictors, tuning is easy and fast.

As to future work:
\bi
\item
 As noted by
F\"{u}rnkranz~\cite{furnkranz05}, data mining is inherently a multi-objective optimization
problem that seeks the smallest model with the highest performance, 
that generalizes best for
future examples (perhaps learned in minimal time using the least amount of data).
In this view, we are using DE to optimize an optimizer. Perhaps a better approach might be
to dispense with the separation of ``optimizer'' and ``learner'' and combine them both
into one system that learns how to tune itself as it executes. If this view is useful,
then instead of adding elaborations to data miners (as done in this paper, or by researchers
exploring hyper-heuristics~\cite{jia2013learning}), it should be possible to radically simplify optimization and data
mining with a single system that rapidly performs both tasks.
\item
Another issue for future work is the implications of these
conclusions to other kinds of software analytics.
 This paper has explored  {\em some} learners using {\em one}  optimizer. Hence, we can make
no claim that DE is the {\em best} optimizer for {\em all} learners.
Rather, our point is that there exists at least some learners
whose performance can be dramatically improved by 
at least one simple optimization scheme.  We hope that this work inspires
much future work as this community develops and debugs best practices for tuning
software analytics.
\ei
 

\section{Acknowledgments}
The work has partially funded by a National Science Foundation CISE CCF award \#1506586.
 
\vspace*{0.5mm}
 
 \scriptsize
\bibliographystyle{plain}

\balance
\bibliography{tuningpredictor}  

   



  


  

\end{document}
 
\subsection{Implications}

time for an end to era of data mining in se? moving on to a new phase of learning-as-optimization

1) learning is actually an optimization tasks (e.g. see fig2 of  learners climbing the roc curve hill in http://goo.gl/x2EaAm)

2) our learners are all contorted to do some tasks X (e.g. minimize expected value of entropy), then we assess them on score Y (recall). which is nuts. maybe we should build the goal predicate into the learner (e.g http://menzies.us/pdf/10which.pdf) 

3) given 1 + 2, maybe the whole paradigm of optimizing param selection is wrong. maybe what we need is a library of bees buzzing around making random choices (e.g. about descritziation) which other bees use, plus their own random choices (e.g. max depth of tree learned from discretized data) which is used by other bees, plus their own random choices (e.g. business users reading the models).  the funky thing here is that it can take some time before some of the bees (the discretizers) get feedback from the community of people using their decision (the tree learners). 




