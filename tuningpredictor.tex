

\documentclass{sig-alternative}
\usepackage{multirow}
\usepackage{color}
\usepackage{graphics}
\usepackage{rotating}
\usepackage{graphics}
\usepackage{colortbl} 
\usepackage{picture}
\usepackage{algorithm}
\usepackage{algorithmicx}
\usepackage{algpseudocode}
\renewcommand{\footnotesize}{\scriptsize}
\definecolor{lightgray}{gray}{0.8}
\definecolor{darkgray}{gray}{0.6}
\renewcommand{\algorithmicrequire}{\textbf{Input:}}
\renewcommand{\algorithmicensure}{\textbf{Output:}}
%%% graph
\newcommand{\crule}[3][darkgray]{\textcolor{#1}{\rule{#2}{#3}}}
\newcommand{\rone}{\crule{1mm}{1.95mm}}
\newcommand{\rtwo}{\crule{1mm}{1.95mm}\hspace{0.3pt}\crule{1mm}{1.95mm}}
\newcommand{\rthree}{\crule{1mm}{1.95mm}\hspace{0.3pt}\crule{1mm}{1.95mm}\hspace{0.3pt}\crule{1mm}{1.95mm}}
\newcommand{\rfour}{\crule{1mm}{1.95mm}\hspace{0.3pt}\crule{1mm}{1.95mm}\hspace{0.3pt}\crule{1mm}{1.95mm}\hspace{0.3pt}\crule{1mm}{1.95mm}} 
\newcommand{\rfive}{\crule{1mm}{1.95mm}\hspace{0.3pt}\crule{1mm}{1.95mm}\hspace{0.3pt}\crule{1mm}{1.95mm}\hspace{0.3pt}\crule{1mm}{1.95mm}}
\newcommand{\quart}[3]{\begin{picture}(100,6)%1
{\color{black}\put(#3,3){\circle*{4}}\put(#1,3){\line(1,0){#2}}}\end{picture}}
\definecolor{Gray}{gray}{0.95}
\definecolor{LightGray}{gray}{0.975}

%% timm tricks
\newcommand{\bi}{\begin{itemize}[leftmargin=0.4cm]}
\newcommand{\ei}{\end{itemize}}
\newcommand{\be}{\begin{enumerate}}
\newcommand{\ee}{\end{enumerate}}
\newcommand{\tion}[1]{\S\ref{sect:#1}}
\newcommand{\fig}[1]{Figure~\ref{fig:#1}}
\newcommand{\eq}[1]{Equation~\ref{eq:#1}}

%% space saving measures

\usepackage[shortlabels]{enumitem} 
\usepackage{times}

\usepackage{url}
\def\baselinestretch{1}


\setlist{nosep}
 \usepackage[font={small}]{caption, subfig}
\setlength{\abovecaptionskip}{1ex}
 \setlength{\belowcaptionskip}{1ex}

 \setlength{\floatsep}{1ex}
 \setlength{\textfloatsep}{1ex}
 \newcommand{\subparagraph}{}

\usepackage[compact,small]{titlesec}
\DeclareMathSizes{7}{7}{7}{7} 
\pagenumbering{arabic}
\setlength{\columnsep}{7mm}

\begin{document}

\conferenceinfo{FSE}{'15 Bergamo, Italy}
\title{Implications of Differential Evolution for  Defect Prediction:
Analytics Without Parameter Tuning Considered Harmful?}
\numberofauthors{1}
\author{
\alignauthor
Wei Fu, Tim Menzies, Vivek Nair\\
       \affaddr{Computer Science, North Carolina State University, Raleigh, USA}\\
       {wfu@ncsu.edu, tim.menzies@gmail.com, vivekaxl@gmail.com}} 


 
\maketitle
\begin{abstract}
When learning
software defect predictors,   a
very simple optimizer called differential evolution needs can quickly
find  good tunings for that learner.
Such tunings can have a dramatic change to the performance of a learner;
e.g. in this paper we show that tuning can alter the precision of
a software defect predictor from 2\% to 98\%.

These results prompt for a change to standard methods in software analytics.
At least for defect prediction, 
it is no longer enough to just run a data miner and present the result
{\em without} first conducting an tuning optimizations study.
The implications for other kinds of software analytics is now an open and pressing questions.

%RQ1: Does tuning affect learners' performance?
%RQ2: How to choose 

\end{abstract}

% A category with the (minimum) three required fields
\vspace{1mm}
\noindent
{\bf Categories/Subject Descriptors:} 
D.2.8 [Software Engineering]: Product metrics;
I.2.6 [Artificial Intelligence]: Induction

\vspace{1mm}
\noindent
{\bf General Terms:} Experimentation, Algorithm

\vspace{1mm}
\noindent
{\bf Keywords:} defect prediction, CART, random forests,
differential evolution,
search-based software engineering.

\section{Introduction}

%\begin{raggedleft}
%{\em ``The  common misunderstanding about science 
%%is that scientists seek and find truth. They don't- 
%they make and test models.''}\\ -- Neil Gershenfeld

%{\em Essentially, all models are wrong, but some are useful."}\\ --George Box
%%
%\end{raggedleft}

One of the ``black arts'' of data mining is setting the tuning
parameters that control  the choices within that data miner.
Such tunings can have a dramatic change to the performance of a learner;
e.g. in this paper we show that tuning can alter the precision of
a software defect predictor from 2\% to 98\%.

Many researchers have applied automatic optimizers to find good tunings.
This paper shows that for at least one software analytics problem
(software defect prediction from static code attributes),   a
very simple optimizer called differential evolution needs only   seconds
to just  few minutes to find good tunings.  

These results   challenge much prior work into defect prediction:
Numerous   prominent papers have claimed 
that learnerX is better than learnerY for some software analytics task~\cite{lessmann2008benchmarking,hall11,me07b}.
For example, a recent IEEE TSE article claimed that the 
CART decision tree learner was far worse than Random Forests for
softwre defect predition~\cite{lessmann2008benchmarking}. 
Such conclusions do not survive tuning.
For example,
after tuning, the worst untuned learner (CART) can out-perform the supposedly
best learner (Random Forests).

Another class of conclusions challenged by this paper are research papers
that use data miners to   show that different attributes are ``best'' for different
data sets and even different goals for the tunings. Hence, it is no longer
supportable to use data mining to argue that certain kinds of attributes 
are better than others; e.g. as  done by~\cite{rahman2013how}. 
For example, in 2013,
Rahman et al
\cite{rahman2013how}  used results from logistic regression to conclude
that code metrics are generally less useful than process
metrics for  predicting defects from static code analysis.  
In the same year, Bell et al. (who also used negative binomial regression) argued that
software defect prediction was not improved by adding attributes relating to the social
structure of programmer teams~\cite{bell2013limited}.

Given the simplicity and effectiveness of differential evolution,
and also given the effects this method has on defect prediction, this
paper  motivates  a change to the entire field of software analytics.
At least for softwre defect prediction,
it is no longer enough to just run a data miner and present the result
{\em without} first conducting an tuning optimizations study.

The implications for other kinds of data mining tasks is unclear.
While this paper reports success with tuning defect predictors,
we were not so successful with
other analytics tasks (software development effort estimation).
Hence, we assert that it is now an open and pressing research question is whether or not  
tuning studies are now   required for research reports on software analytics.


\section{Background}
\subsection{Defect Prediction}


This section is our standard introduction to defect prediction~\cite{me15:book1},
plus   some new results from Rahman et al.'s   2014 FSE paper~\cite{rahman14:icse}. 
 


\begin{figure*}[!t]
\renewcommand{\baselinestretch}{0.8}\begin{center}
{\scriptsize
\begin{tabular}{c|l|p{4in}}
amc & average method complexity & e.g. number of JAVA byte codes\\\hline
avg\_cc & average McCabe & average McCabe's cyclomatic complexity seen
in class\\\hline
ca & afferent couplings & how many other classes use the specific
class. \\\hline
cam & cohesion amongst classes & summation of number of different
types of method parameters in every method divided by a multiplication
of number of different method parameter types in whole class and
number of methods. \\\hline
cbm &coupling between methods &  total number of new/redefined methods
to which all the inherited methods are coupled\\\hline
cbo & coupling between objects & increased when the methods of one
class access services of another.\\\hline
ce & efferent couplings & how many other classes is used by the
specific class. \\\hline
dam & data access & ratio of the number of private (protected)
attributes to the total number of attributes\\\hline
dit & depth of inheritance tree &\\\hline
ic & inheritance coupling &  number of parent classes to which a given
class is coupled (includes counts of methods and variables inherited)
\\\hline
lcom & lack of cohesion in methods &number of pairs of methods that do
not share a reference to an instance variable.\\\hline
locm3 & another lack of cohesion measure & if $m,a$ are  the number of
$methods,attributes$
in a class number and $\mu(a)$  is the number of methods accessing an
attribute,\newline
then
$lcom3=((\frac{1}{a} \sum_j^a \mu(a_j)) - m)/ (1-m)$.
\\\hline
loc & lines of code &\\\hline
max\_cc & maximum McCabe & maximum McCabe's cyclomatic complexity seen
in class\\\hline
mfa & functional abstraction & number of methods inherited by a class
plus number of methods accessible by member methods of the
class\\\hline
moa &  aggregation &  count of the number of data declarations (class
fields) whose types are user defined classes\\\hline
noc &  number of children &\\\hline
npm & number of public methods & \\\hline
rfc & response for a class &number of  methods invoked in response to
a message to the object.\\\hline
wmc & weighted methods per class &\\\hline
\rowcolor{lightgray}
defect & defect & Boolean: where defects found in post-release bug-tracking systems.
\end{tabular}
}
\end{center}
\caption{OO measures used in our defect data sets.  Last line is
the dependent attribute (whether a defect is reported to  a
post-release bug-tracking system).}\label{fig:ck}
\end{figure*}



Human programmers are clever, but flawed. Coding  adds functionality, but also defects.
Hence, software sometimes crashes (perhaps at the most awkward or dangerous moment) or delivers
the wrong functionality. For a very long list of software-related errors,
see  Peter Neumann's ``Risk Digest'' at catless.ncl.ac.uk/Risks.

Since programming inherently
introduces defects into  programs, it is important to test them before they're used.
Testing is expensive.
Software assessment budgets are finite
while assessment effectiveness increases 
exponentially with assessment effort.
For example, for  black-box testing methods,
a {\em linear} increase
in the confidence $C$ of finding  defects
can take {\em exponentially} more effort\footnote{A randomly selected 
input to a program will find a fault with probability $p$.
After $N$ random black-box tests, the chances of the inputs 
not revealing any fault 
is $(1-p)^N$. Hence, the chances $C$ of seeing the fault is $1-(1-p)^N$
which can be rearranged to 
 $N(C,p)=log(1 -
C)/log(1-p)$. For example, $N(0.90,10^{-3})=2301$
but $N(0.98,10^{-3})=3901$; i.e. nearly double the number of tests.}.
Exponential costs quickly exhaust finite resources so
standard practice is to apply the best
available  methods on code sections that seem   most critical. 
But 
any method that focuses on parts of the code
can blind us to defects in other areas. Some  {\em
lightweight sampling policy} should be used to explore the rest of the system.  This
sampling policy will always be incomplete.
Nevertheless, it is the only option when
resources prevent a complete assessment of everything.

One such lightweight sampling policy is defect predictors learned from static code attributes.
Given software described in the attributes of \tab{ck},   data miners can
learn where the probability of software defects is highest.

The rest of this section argues that such defect predictors are   {\em easy to
use}, {\em widely-used}, and {\em useful} to use.

{\em Easy to use:} Static code attributes can be automatically collected, even for very large systems~\cite{nagappan05}.
Other methods, like  manual code reviews, are far slower and far more labor-intensive.
For example, depending on the review methods, 8 to 20 LOC/minute can be
inspected and this effort repeats for all members of the review team,
which can be as large as four or six people~\cite{me02f}. 

{\em Widely used:}  Researchers and industrial practitioners  use static attributes to guide software 
quality predictions.
 Defect prediction models have been reported
  at Google~\cite{lewis13}.
Verification and validation (V\&V) textbooks
(\cite{rakitin01}) advise using static code complexity attributes
to decide which modules are worth manual inspections.  


{\em Useful:}
Defect predictors often  find the location of  70\% (or more)
of the defects in code~\cite{me07b}.
Defect predictors have some level of generality:
predictors learned at NASA~\cite{me07b} have also been found useful elsewhere
(e.g. in Turkey~\cite{tosun10,tosun09}.
The success of this method in  predictors in finding bugs is   markedly
higher than other currently-used
industrial
methods such as manual code reviews. For example, 
a  panel at {\em IEEE Metrics
2002}~\cite{shu02} concluded that manual software  reviews can find ${\approx}60\%$ 
of defects.
In other work, 
Raffo documents the typical    defect detection capability of
industrial review methods:   around 50\%
 for full Fagan inspections~\cite{fagan76} to
21\% for less-structured inspections.

Not only do static code defect predictors perform well compared to manual methods,
they also are competitive with certain automatic methods.
A recent study at ICSE'14, Rahman et al.~\cite{rahman14:icse} compared
(a) static code analysis tools FindBugs, Jlint, and Pmd and (b)
static code defect predictors
(which they called ``statistical defect prediction'') built using logistic regression.
They found  no significant differences in the cost-effectiveness
of these  approaches. Given this equivalence, it is significant to note that 
static code defect prediction can be quickly adapted to new languages by building lightweight
parsers that find   information like \tab{ck}. The same is not true for   static code analyzers-- these need  extensive modification before they can be used on new
languages.



 

\subsection{Data Mining}
Static code defect prediction applies data miners to build its predictors. How does data mining work?

This section describes this study's learners (CART~\cite{brieman00}, Random Forest~\cite{breiman84}, 
and WHERE~\cite{menzies2013local}) and their
tuning parameters (summarised in \fig{parameters}).

Our implementations
for CART and Random Forest comes from 
SciKitLearn~\cite{scikit-learn}.
WHERE is available from
github.com/ai-se/where\footnote{\wei{ need a nice  where repo. with an
data and code examples sub-directory. clean code. throw out anything not needed. before april1.}}.
WHERE has the most tuning parameters. This turns out
to be important since, in the experiments shown below, tuned WHERE 
out-performed the other learners-- a result suggesting that
if you are going to tune learners, then use one with many tuning options.

%%%%%%%%%%%%%%%% list of parameters%%%%%%%%%%%%%%%%%%%%%
\renewcommand\arraystretch{1.2}
\begin{figure*}[t!]
\scriptsize
  \centering
	\begin{tabular}{|c|c|c|c|l|}
	\cline{1-5}
	\begin{tabular}[c]{@{}c@{}}Learner \\ Name\end{tabular} & Parameters & Default &\begin{tabular}[c]{@{}c@{}}Tuning\\ Range\end{tabular}& 
\multicolumn{1}{c|}{Description} \\ \hline
	\multirow{8}{*}{\begin{tabular}[c]{@{}c@{}}Where-based\\ Learner\end{tabular}} 
	& threshold & 0.5 &[0.01,1]& The value to determine defective or not .\\ \cline{2-5} 
	& infoPrune & 0.33 &[0.01,1]& The percentage of features to consider  for the best 
split to build CART tree\footnote{Since the Where-based learner will build two trees, the first 
one is for clustering and the second one is building prediction model. we explicitly call Where-
clustering tree and CART tree, respectively}. \\ \cline{2-5} 
	 & min\_sample\_split & 4& [1,10]& The minimum number of samples required to split an internal node of
CART tree. \\ \cline{2-5} 
	 & min\_Size & 0.5 &[0.01,1]& \begin{tabular}[c]{@{}l@{}}The value to determine the minimum 
number of samples to be a Where-clustering tree \\ based on  ${n\_samples}^ {min\_Size}$.
\end{tabular} \\ \cline{2-5} 
    & wriggle & 0.2 &[0.01, 1] & The threshold to determine which branch in  Where tree to be pruned\\ \cline{2-5}
	 & depthMin & 2 & [1,6]&The minimum depth of the tree below which no pruning for Where-
clustering tree. \\ \cline{2-5} 
	 & depthMax & 10 &[1,20]& The maximum depth of the Where-clustering tree. \\ \cline{2-5} 
	 & wherePrune & False &T/F& Whether or not to prune the Where-clustering tree. \\ \cline{2-5}
	 & treePrune & True &T/F& Whether or not to prune the classification tree built by CART. \\ \cline{2-5} 
\hline
\multirow{4}{*}{CART} & threshold & 0.5 &[0,1]& The value to determine defective or not. \\ \cline{2-5} 
	 & max\_feature & None &[0.01,1]& The number of features to consider when looking for the best 
split. \\ \cline{2-5} 

	 & min\_sample\_split & 2 &[2,20]& The minimum number of samples required to split an 
internal node. \\ \cline{2-5} 
	 & min\_samples\_leaf & 1 & [1,20]&The minimum number of samples required to be at a leaf 
node. \\ \cline{1-5}  
       \multirow{5}{*}{\begin{tabular}[c]{@{}c@{}}Random \\ Forests\end{tabular}}  & threshold & 0.5 & [0.01,1] & The value to determine defective or not. \\ 
\cline{2-5} 
	 & max\_feature & None &[0.01,1]& The number of features to consider when looking for the best 
split. \\ \cline{2-5} 
	 & max\_leaf\_nodes & None &[1,50]& Grow trees with max\_leaf\_nodes in best-first fashion. \\ \cline{2-5} 
	 & min\_sample\_split & 2 &[2,20]& The minimum number of samples required to split an 
internal node. \\ \cline{2-5} 
	 & min\_samples\_leaf & 1 &[1,20]&The minimum number of samples required to be at a leaf 
node. \\ \cline{2-5} 
	 &  n\_estimators & 100 & [50,150]&The number of trees in the forest.\\ \cline{2-5}
	 \hline

	\end{tabular}
    \caption {List of parameters to be tuned.}
\label{fig:parameters}
\end{figure*}
 
\subsection{Why Study These Algorithms?}


This paper studies WHERE since this was the first learner we tried to tune and, as shown below,
it offers an interesting case study on the benefits of tuning.

This paper also studies  CART and Random Forest since  these were used in 
a recent IEEE TSE paper by Lessmann et al.~\cite{lessmann2008benchmarking} that compared 21 different 
learners for software defect prediction:
\bi
\item
{\em Statistical classifiers:}
Linear    discriminant analysis,
Quadratic discriminant analysis,
Logistic regression,
Naive Bayes,
Bayesian networks,
Least-angle regression,
Relevance vector machine,

\item
{\em Nearest neighbor methods:}
k-nearest neighbor,
K-Star

\item
{\em Neural networks:}
Multi-Layer Perceptron,
Radial bias function network,

\item
{\em Support vector machine-based classifiers:}
Support vector machine,
Lagrangian SVM
Least squares SVM,
Linear programming,
Voted perceptron,

\item
{\em Decision-tree approaches:}
C4.5 decision tree,
CART,
Alternating decision tree.
\item
{\em Ensemble methods:}
Random Forest,
Logistic Model Tree.
\ei
In that study, CART was severely trounced (was ranked last) and Random Forest was
the standout best method. The experiments shown below both confirm and refute
that ranking. In a result consistent with the prior result, untuned Random Forest performs best.
However, after tuning, the worst learner found by Lessmann et al. (CART) performed better
than Random Forest.
  

\subsection{Learners and Their Tunings}

CART, Random Forest, and WHERE are all  tree learners that divide a data set, then recurse
on each split.
If data contains more than {\em min sample split}, then a split is attempted.
On the other hand, if a split contains no more than {\em min samples leaf}, then recursion stops. For CART and Random Forest use a 
user-supplied constant for this parameter while
WHERE computes $m$={\em min samples leaf} from the size of the data
sets via  $m=\mathit{size}^\mathit{min size}$ (so, for WHERE,
{\em min size} is the parameter to be tuned).

These learners
generate numeric predictions which are converted
into binary ``yes/no'' decisions via \eq{yesno}. Hence, they all use the {\em threshold} value $T$ discussed in \tion{eg}.

These learners use different techniques to explore the splits:
\bi
\item
CART finds the attributes whose ranges contain rows with least variance in the number
of defects\footnote{If an attribute ranges $r_i$ is found in 
$n_i$ rows each with a  defect count variance of $v_i$, then CART seeks the attributes
whose ranges minimizes $\sum_i \left(\sqrt{v_i}\times n_i/(\sum_i n_i)\right)$.}.
\item
Random Forest    divides data like CART,
but it builds $F>1$  trees, each time with a subset of
the attributes (selected at random). 
\item
WHERE projects the data on to a dimension it synthesizes from the raw data using
a process analogous to principle component analysis\footnote{
PCA  synthesises  new
attributes $e_i, e_2,...$
that extends across the dimension of greatest  variance in the data  with attributes $d$.  
This process  combines
redundant  variables into a smaller set of variables  (so $e \ll d$) since those
redundancies become (approximately) parallel lines
in $e$ space. For all such redundancies \mbox{$i,j \in d$}, we 
can ignore $j$ 
since effects that change over $j$ also
change in the same way over $i$.
PCA is also useful for skipping over noisy variables from $d$-- these
variables are effectively ignored since    they  do not contribute to the variance in the data.}.
WHERE   divides  at the median point of that projection. On recursion,
this generates a dendogram, the leaves of which are clusters of  very similar examples.
\ei
WHERE's {\em infoPrune} tuning parameter then choices the
attributes   that best select  different clusters.
WHERE pretends its clusters are ``classes'', then 
asks the InfoGain of the
Fayyad-Irani discretizer~\cite{FayIra93Multi}, to rank the attriubutes.
WHERE then ignores everything except the top   {\em infoPrune} percent of the sorted
attributes.

Optionally, if the {\em where prune} option is set, 
WHERE  continues to applies infogain criteria  recursively to build a tree that selects for the
different clusters. If WHERE's {\em tree prune} parameter is enabled, then WHERE also prunes  superfluous sub-trees. For example, if a sub-tree and its parent have the same 
majority cluster
(one that occurs most frequently), then we prune the sub-tree.
This tree pruning  sometimes
prunes aways all  cluster selectors branches. To tame this effect, the {\em wriggle} parameter
blocks tree pruning for at least the first {\em wriggle} number of initial branches.

Other tuning parameters are learner specific. For example,
{\em max feature} is used by
CART and Random Forest to select the number of attributes
used to build one tree.
CART's default is to use all the attributes while 
Random Forest usually selects the square root of the number
of attributes.
Also,
  {\em max leaf nodes} is the upper bound on leaf notes generated in a 
  Random Forest.



\subsection{Tuning Algorithms}


 \subsubsection{Parametric Tuning Algorithms}
The  goal of this paper is to adjust the tuning parameters of \fig{parameters}
in order to   optimize (improve) some particular performance scores
generated by a particular learner being applied to  a particular data set.
For this task, we do not use traditional parametric numeric optimizer  
such as  gradient descent optimizers~\cite{saltelli00} that require models comprise
differential functions (i.e. functions of real-valued variables whose derivative exists at each point in its domain).
This is impractical  for  our learners since their internal states are   not a smoothly differential continuous function.
Rather, learners being tuned  contains many regions with many different properties (tuning options can
drive the learner into very different modes with very different performance properties).


 \subsubsection{Non-Parametric Tuning Algorithms}
 
Non-parametric  optimizers   make no assumption
about the model being only differential functions. One such optimizer
is simulated annealing. SA generates {\em new} solutions
 by randomly perturbing (a.k.a. ``mutating'') some part of an {\em old}
 solution.  {\em New} replaces {\em old} if (a) it scores higher; or
 (b) it reaches some probability set by a ``temperature'' constant. Initially,
 temperature is high so SA jumps to sub-optimal solutions (this allows
 the algorithm to escape from local minima). Subsequently, the
 ``temperature'' cools and SA only ever moves to better {\em new}
 solutions. 
 SA is often used in search-based SE
 e.g.~\cite{fea02a,me07f}, perhaps due to its simplicity.

SA was invented   in the 1950s, when
 computer RAM was very small~\cite{kirkpatrick83}. A standard SA algorithm needs
 only space for three solutions {\em new, old} and the {\em best} seen so far.
  In the 1960s, when more RAM became available, it became standard to
 generate many {\em new} mutants, and then combine together parts of
 promising solutions~\cite{goldberg79}.  Such {\em evolutionary
   algorithms} (EA) work in {\em generations} over a population of
 candidate solutions.  Initially, the population is created at random.
 Subsequently, each generation makes use of select+crossover+mutate
 operators to pick promising solutions, mix them up in some way, and
 then slightly adjust them.
 EAs are also often used in search-based
 software engineering, particularly in test case generation~\cite{andrews07,andrews10}
 or refactoring~\cite{Weimer:2009}

 Later work focused on creative ways to control the
 mutation process. Tabu search and scatter search
 work to bias new mutations away from prior
 mutations~\cite{Glover1986563,Beausoleil2006426,Molina05sspmo:a,4455350}.
 Particle swarm
 optimization randomly mutates multiple solutions
 (which are called ``particles''), but biases those
 mutations towards the best solution seen by one
 particle and/or by the neighborhood around that
 particle~\cite{pan08}.
 Differential evolution mutates solutions by
 interpolating between members of the current
 population~\cite{storn1997differential}.  
 
Another more recent technique that has claimed much attention
are   heuristics that decompose the total space into many smaller problems, and then which use a simpler optimizer for each region. 
For example, in $\mathcal{E}$-domination~\cite{deb05}, the  user is asked
`what is the lower threshold $\mathcal{E}$ on the size of a useful effect?''. The solution space
is then divided into boxes of size $\mathcal{E}$ and linked such that  the  set $X.\mathit{lower}$ contains boxes with worse objective scores that $X$.  Solutions in pairs boxes are  quickly compared  using   small samples from each  and, if some box $X$ is found to be inferior, then it is quickly pruned along with all
solutions in the $X.\mathit{lower}$ boxes.
Later research generalized this approach. MOEA/D (multiobjective
evolutionary algorithm based on decomposition~\cite{zhang07}) is a generic framework that decomposes a multiobjective optimization problem into many smaller single problems, then applies a second optimizer to each smaller subproblem, simultaneously.   Other work in this arena are the response surface methods
that quickly find multiple approximations to the problem, each of which holds for a very tiny region.
Each region has a ``slope'' and examples in that region are pushed along the slope towards better
solutions~\cite{krall15,Zuluaga:13}.
 
 
\input{algo} 
 \subsubsection{Selecting a Tuning Algorithm}
 
From all the above methods, how do we select which optimizers to apply to tuning data miners.
Cohen~\cite{cohen95} advises comparing any supposedly more
sophisticated method against the simplest possible alternative. For
example, in one study with ``floor effects'', Holte showed that,
often, much of the performance of complex multi-level decision trees
could be easily achieved using a much simpler single-level decision
tree learner called 1R~\cite{holte93}. He therefore recommends a very simple rule learner
(called ``1R'') as a
kind of ``scout'' that can do a quick preliminary analysis of a data
set and which can report back if that data really requires a more
complex analysis.

To find our ``scout'',  we used engineering judgement to sort  the above algorithms from simplest to most complex.
The three simplest optimizers are SA, $\mathcal{E}$-domination, and 
differential evolution (each can be coded in less than a page of some high-level scripting language). Our reading of the current literature is that there are more  advocates for
differential evolution than
  SA or $\mathcal{E}$-domination:
  \bi
  \item
  When the MOEA/D community requires a secondary optimizer, they often use  differential evolution~\cite{zhang07,5583335}.
  \item
 Vesterstrom and Thomsen~\cite{Vesterstrom04} report that DE is competitive with 
   particle swarm optimization and a genetic algorithm. 
   \ei
DEs have been applied before for   parameter tuning (e.g. see~\cite{omran2005differential, chiha2012tuning}) but this is the first time they have been applied to
optimizing defect prediction from static code attributes.  


 
 


 
 


\subsection{Differential Evolution Algorithm}
Differential Evolution (DE)\cite{storn1997differential} is a stochastic search algorithm that 
optimizes a problem by iteratively trying to improve a  population of candidate solutions with 
regard to a given quality measurement. Such method makes no assumptions about the 
problem being optimized and has already been used as a parameter tuner
\cite{omran2005differential, chiha2012tuning}.

DE starts with creating a population of candidate solutions and then generates new 
candidates based on $New = X+f*(Y-Z)$, where $X$, $Y$ and $Z$ are randomly selected 
solutions from the current frontier and $f$ is a crossover factor. The newly generated solution 
will be added into the next generation of solutions if dominating previous old points in the 
frontier. To avoid meaning less iteration, early termination strategy is applied that is at the 
beginning, assign a value to the $life$ parameter, it would be reduced by one each time   
when the new generation of candidate solutions does not improve in terms of quality 
measurement. The DE will stop when the $life$ equals to 0.

Algorithm \ref{alg:DE} is a list of pseudocode of DE with early termination for maximizing a 
score function, where $np$ is the number of population in each generation, $f$ is the 
crossover factor as mentioned, $cf$ is the probability for crossover operation to generate new 
candidate, $life$ is to control termination.

\begin{algorithm}
\caption{Pesudocode for DE with Early Termination}
\label{alg:DE}
\begin{algorithmic}[1]
\Require $np = 10$, $f=0.75$, $cr=0.3$, $life = 5$
\Ensure $S_{best}$
     
     \Function{DE}{$np$, $f$, $cr$, $life$}
 \State $Population  \gets $ InitializePopulation($np$)
 \State $S_{best} \gets $GetBestSolution($Population $)
 \While{$life > 0$}
\State $NewGeneration \gets \emptyset$
\For{$i=0 \to np-1$}
\State $S_i \gets$ GenNew($Population [i], Population , cr, f$)
\If {Score($S_i$) >Score($Population [i]$)}
\State $NewGeneration$.append($S_i$)
\Else
\State $NewGeneration$.append($Population [i]$)
\EndIf
\EndFor
\State $Population  \gets NewGeneration$
\If{$\neg$ Improve($Population $)}
\State $life -=1$
\EndIf
\State $S_{best} \gets$ GetBestSolution($Population $)
 \EndWhile
\State \Return $S_{best}$
\EndFunction
\Function{Score}{$Candidate$}
   \State set tuned parameters according to $Candidate$
   \State TrainLearner()
   \State $scores \gets$TestLearner()
   \State \% scocres = [pd, f, precision]
   \State $s \gets scores[i] $ 
   \State \% i determines the goal of tuning
   \State \Return$s$
\EndFunction
\Function{GenNew}{$old, pop, cr, f$}
  \State $a, b, c\gets threeOthers(pop,old)$ 
  \State \% pick 3 other points from the population
  \State $newf \gets \emptyset$
  \For{$i=0 \to np-1$}
       \If{$cr < random()$}
         \State $newf$.append($old[i]$)
                \Else
                  \If{typeof(old[i]) == bool}
                    \State $newf$.append(not $old[i]$)
         \Else
          \State $newf$.append(trim($i$,($a[i] + f * (b[i] - c[i]$))))
          \State \%trim() will make new value fall in the range
         \EndIf
       \EndIf
  \EndFor
\EndFunction
        \end{algorithmic}            
\end{algorithm}

%\begin{algorithm}
%\begin{algorithmic}[1]
% \KwData{this text}
% \KwResult{how to write algorithm with \LaTeX2e }
% initialization\;
% \While{not at end of this document}{
%  read current\;
%  \eIf{understand}{
%   go to next section\;
%   current section becomes this one\;
%   }{
%   go back to the beginning of current section\;
%  }
% }
% \caption{How to write algorithms}
% \end{algorithmic}
%\end{algorithm}

\section{Experiment}

\subsection{Data Set}

The data used in this study is from PROMISE repository. Ten software defect predicition data 
sets are analyzed. They're {\it ant}, {\it camel}, {\it ivy}, {\it jedit}, {\it log4j}, {\it lucene}, {\it 
synapse}, {\it velocity}, {\it xalan} and {\it xerces}. Each of these data sets is composed of 
several software modules with number of defects and code attributes. For more detailed 
description of code attributes and the original data sets, please refer to  http://openscience.us/
repo/

\subsection{Experiment Design}

The experiment aims at investigate whether tuning helps learners improve performance in 
terms of accuracy measurement. We choose compare WHERE-based learner with and 
without tuning parameters and two {\it significantly different} learners Random Forest and 
CART according to \cite{lessmann2008benchmarking}. As mentioned above, we'd like to see 
whether tuning will help change the rank of CART and make it comparable with Random 
Forest. 

To evaluate accuracy performance of learners, several measurements are proposed, like 
probability of  detection({\it pd}) and probability of false alarm ({\it pf})\cite{menzies2007data}, 
the area under the receiver operating characteristics curve ({\it AUC})
\cite{lessmann2008benchmarking}, and precision\cite{zhang2007comments}. In this work, we 
expect learners should identify as many defective modules as possible while avoiding false 
alarm. Therefore, learners are evaluated by both of {\it pd} and {\it pf} simaltanieously. A single 
measure, G-measure, defined as the harmonic mean of {\it pd} and $1-{\it pf}$  is used. The 
G-measure value is between 0 and 1. The higher, the better.
\begin{equation}
G = \frac{2*(1-pf)*pd}{1-pf+pd}
\end{equation}

In this experiment, we use three different portions of one project data set for training, tuning 
and testing process. In contrast to hold out way used in \cite{lessmann2008benchmarking, 
menzies2007data}, we separate the data sets in order. Since learners are designed to predict 
defects in future projects,  any randomly data set selection without taking into the time series 
will not sufficient to evaluate the performance of predicting future. To the most, that is good to 
evaluate the accuracy of classification but not predicting future. Since we have 10 different 
project data, each of which contains least 3 evolutionary versions. We use the following policy 
to select the data: in each project, we only use the last three data files for experiment. 
Specifically, the $nth$,  $(n-1)th$, $(n-2)th$ versions of project data are selected for testing, 
tuning and training learners, respectively. This will make sure that we don't use the future 
project data to train learners and predict previous project.

To investigates the impacts of parameters on learners, we use DE as the tuner and compare the G-measure 
values of Where-based learner with and without tuning, CART with and without tuning and Random 
Forests. Since the tuning time for Random Forests is very long, hopefully other researchers design
new heuristics to speed up the tuning process for Random Forests.  For the time being, even though
we don't tune Random Forests, if tuning help CART outperform Random Forests or improve itself performance ,
we still could conclude that tuning is helpful and necessary when comparing learners.

Besides the Where-based learner implemented by ourself,  we use the CART and Random Forest modules 
from scikit-learn \cite{scikit-learn} for this experiment. The parameters associated with different learners are listed in Fig.\ref{fig:parameters}. (\textbf{NEED to elaborate that there're different versions of CART, what's the point to do this experiment}). For each data set, run CART, naive Where-based learner and Random Forests with corresponding default values. Then using DE to tune corresponding parameters for CART and Where-based Learner,  and run them again with the optimal parameters from tuning process to test the performance. 
This study ranks learners using the Scott-Knott procedure recommended by Mittas \& Angelis in their 2013
IEEE TSE paper~\cite{mittas2013ranking}.  This method sorts a list of $l$ treatments with $ls$ measurements by their median score. It then splits $l$ into sub-lists $m,n$ in order to maximize the expected value of differences  in the observed performances before and after divisions. 

\section{Result}

\section{Discussion}

\section{Related Work}

Tuning in efforts estimation, software engineering.

Defect Prediction



\section{Threats to Validity}

Internal and external threats

\section{Conclusion}

\section{Acknowledgments}

\bibliographystyle{unsrt}
\bibliography{tuningpredictor}  




%%%%%%%%%%%%%%%% list of parameters%%%%%%%%%%%%%%%%%%%%%
\renewcommand\arraystretch{1.2}
\begin{figure*}[t!]
\renewcommand{\baselinestretch}{0.8}
\scriptsize
  \centering
\begin{tabular}{|c|c|c|c|l|}
\cline{1-5}
\begin{tabular}[c]{@{}c@{}}Learner \\ Name\end{tabular} & Parameters & Default &\begin{tabular}[c]{@{}c@{}}Tuning\\ Range\end{tabular}& 
\multicolumn{1}{c|}{Description} \\ \hline
\multirow{8}{*}{\begin{tabular}[c]{@{}c@{}}Where-based\\ Learner\end{tabular}} 
& threshold & 0.5 &[0.01,1]& The value to determine defective or not .\\ \cline{2-5} 
& infogain & 0.33 &[0.01,1]& The percentage of features to consider  for the best 
split to build CART tree\footnote{Since the Where-based learner will build two trees, the first 
one is for clustering and the second one is building prediction model. we explicitly call Where-
clustering tree and CART tree, respectively}. \\ \cline{2-5} 
& min\_sample\_size & 4& [1,10]& The minimum number of samples required to be a leaf for 
CART tree. \\ \cline{2-5} 
& min\_Size & 0.5 &[0.01,1]& \begin{tabular}[c]{@{}l@{}}The value to determine the minimum 
number of samples to be a Where-clustering tree \\ based on  ${n\_samples}^ {min\_Size}$.
\end{tabular} \\ \cline{2-5} 
    & wriggle & 0.2 &[0.01, 1] & The threshold to determine which branch in  Where tree to be pruned\\ \cline{2-5}
& depthMin & 2 & [1,6]&The minimum depth of the tree below which no pruning for Where-
clustering tree. \\ \cline{2-5} 
& depthMax & 10 &[1,20]& The maximum depth of the Where-clustering tree. \\ \cline{2-5} 
& wherePrune & False &T/F& Whether or not to prune the Where-clustering tree. \\ \cline{2-5}
& treePrune & True &T/F& Whether or not to prune the classification tree built by CART. \\ \cline{2-5} 
\hline
\multirow{4}{*}{CART} & threshold & 0.5 &[0,1]& The value to determine defective or not. \\ \cline{2-5} 
& max\_feature & None &[0.01,1]& The number of features to consider when looking for the best 
split. \\ \cline{2-5} 

& min\_sample\_split & 2 &[2,20]& The minimum number of samples required to split an 
internal node. \\ \cline{2-5} 
& min\_smaples\_leaf & 1 & [1,20]&The minimum number of samples required to be at a leaf 
node. \\ \cline{1-5}  
       \multirow{5}{*}{\begin{tabular}[c]{@{}c@{}}Random \\ Forests\end{tabular}}  & threshold & 0.5 & [0.01,1] & The value to determine defective or not. \\ 
\cline{2-5} 
& max\_feature & None &[0.01,1]& The number of features to consider when looking for the best 
split. \\ \cline{2-5} 
& max\_leaf\_nodes & None &[1,50]& Grow trees with max\_leaf\_nodes in best-first fashion. \\ \cline{2-5} 
& min\_sample\_split & 2 &[2,20]& The minimum number of samples required to split an 
internal node. \\ \cline{2-5} 
& min\_smaples\_leaf & 1 &[1,20]&The minimum number of samples required to be at a leaf 
node. \\ \cline{2-5} 
&  n\_estimators & 100 & [50,150]&The number of trees in the forest.\\ \cline{2-5}
\hline

\end{tabular}
    \caption {List of parameters to be tuned in Where-based learner and CART in scikit-learn.}
\label{fig:parameters}
\end{figure*}

%%%%%%%%%%%%%%%% optimal parameters from tuning process%%%%%%%%%%%%%%%%%%%%%
%\begin{figure*}[t!]
%\scriptsize
%  \centering
%	\begin{tabular}{|c|c|c|c|c|c|c|c|c|c|c|c|c|c|c|}
%	\hline
%	\begin{tabular}[c]{@{}c@{}}Learner \\ Name\end{tabular}&Parameters  & Default & 
%ant  & camel &ivy & jedit & log4j & lucene & poi & synapse & velocity & xalan & xerces \\ \hline
%	\multirow{8}{*}{\begin{tabular}[c]{@{}c@{}}Where-based\\ Learner\end{tabular}} 
%	& threshold & 0.5 & 0.16 & 0.87 &0.89 & 0.68 &0.58 &0.62&0.06&0.09&1&0.64 &0.02\\ \cline{2-14} 
%	& infogain & 0.33 & 0.33 & 0.49 &1&0.82 &0.12 &0.54&0.38&0.01&1&0.36&0.25\\ \cline{2-14} 
%	& min\_sample\_size & 4& 8 &1 &1&5&8&1&5&9&1&5&5\\ \cline{2-14} 
%	& min\_Size & 0.5& 1 & 0.67 & 0.99&1&0.63&1&0.95&0.59&1&0.88 &0.64\\ \cline{2-14} 
%	& wriggle &
%	& depthMin & 2 & 3 & 4 &1&1&2&1&1&5&1&1&3\\ \cline{2-14} 
%	& depthMax & 10 & 8 &14 &16&18&1&16&13&5&9&18&20\\ \cline{2-14} 
%	& wherePrune & False & True & True &False&False&False&True&True&False&True&False &True\\ \cline{2-14} 
%	& treePrune & True& False & True &True&True&False&False&True&False&False&False &False\\ 
%\hline
%	\multirow{4}{*}{CART} 
%	& threshold & 0.5 & 0.01 & 0.98 &0.94&0.67&0.47&1&0.01&0.12&0.9&0.57 &0.01\\ \cline{2-14} 
%	& max\_feature & None &0.01 & 1 &0.29&0.28&0.53&0.13&0.01&0.01&0.87&0.47 &0.01\\ \cline{2-14} 
%	%& max\_depth & None &27 & 46 &33&21&42&36&39&17&29&11&31\\ \cline{2-14} 
%	& min\_sample\_split & 2 &6 &2 &13&17&12&2&14&12&7&16 &13\\ \cline{2-14} 
%	& min\_smaples\_leaf & 1 &6 &3 &18&12&3&10&4&20&8&11&1\\ \hline  
%    \multirow{6}{*}{\begin{tabular}[c]{@{}c@{}}Random \\ Forests\end{tabular}}  
%    & threshold & 0.5 & 0.06 &0.88 &1&0.73&0.41&0.81&0.11&0.16&1&0.63&0.29\\ \cline{2-14} 
%	& max\_feature & None & 0.21 &0.98 &0.78&0.73&0.36&0.35&0.01&0.01&0.01&0.65 &0.89\\ \cline{2-14} 
%	& max\_leaf\_nodes & None & 31 &35 &41&40&23&12&26&41&46&49&35\\ \cline{2-14} 
%	& min\_sample\_split & 2 &12 &14 &2&5&8&11&1&16&1&9&20\\ \cline{2-14} 
%	& min\_smaples\_leaf & 1 &6 &15 &2&17&3&9&19&8&2&14&9\\ \cline{2-14} 
%	&  n\_estimators & 100 &111 &120 &89&68&64&84&107&100&79&113&50\\ \hline
%
%	\end{tabular}
%    \caption {Optimal parameters from tuning process with objetive of G measure over different data sets.}
%\label{fig:parameters}
%\end{figure*}

%%%%parameters for pd %%%%%%
\begin{figure*}[!t]

\renewcommand{\baselinestretch}{0.8}
\resizebox{\textwidth}{!}{
\scriptsize
\centering
  \begin{tabular}{|c |c |c |c |c |c |c |c |c |c |c |c |c |c |c |c |c |c |c |c |}
    \hline
  \begin{tabular}[c]{@{}c@{}}Learner \\ Name\end{tabular}&Parameters  & Default &antV0&antV1&antV2&camelV0&camelV1&ivyV0&jeditV0&jeditV1&jeditV2&log4jV0&luceneV0&poiV0&poiV1&synapseV0&velocityV0&xercesV0&xercesV1\\ 
 \hline
\multirow{8}{*}{\begin{tabular}[c]{@{}c@{}}Where\\based\\ Learner\end{tabular}}
& threshold& 0.5& 0.17& 0.17& 0.02& 0.04& 0.42& 0.7& 0.8& 0.25& 0.32& 0.13& 0.61& 0.77& 0.09& 0.02& 0.68& 0.17& 0.01\\ \cline{2-20}
& infoPrune& 0.33& 0.13& 0.06& 0.12& 0.35& 0.44& 0.26& 0.33& 0.81& 0.03& 0.32& 0.89& 0.05& 0.1& 0.97& 0.15& 0.24& 0.01\\ \cline{2-20}
& min\_sample\_size& 4& 5& 2& 4& 1& 7& 6& 9& 6& 9& 8& 1& 3& 3& 9& 5& 4& 2\\ \cline{2-20}
& min\_Size& 0.5& 0.49& 0.24& 0.41& 0.26& 0.12& 0.03& 0.26& 0.06& 0.19& 0.38& 0.02& 0.07& 0.2& 0.2& 0.32& 0.13& 0.86\\ \cline{2-20}
& wriggle& 0.2& 0.65& 0.19& 0.42& 0.37& 0.22& 0.56& 0.64& 0.26& 0.79& 0.97& 0.23& 0.52& 0.99& 0.78& 0.15& 0.58& 0.55\\ \cline{2-20}
& depthMin& 2& 4& 3& 2& 5& 5& 4& 3& 5& 4& 4& 2& 4& 2& 1& 4& 3& 1\\ \cline{2-20}
& depthMax& 10& 16& 7& 19& 11& 14& 16& 12& 12& 19& 17& 12& 13& 15& 13& 3& 17& 15\\ \cline{2-20}
& wherePrune& False& True& False& False& False& False& False& False& True& True& False& False& True& False& True& False& False& False\\ \cline{2-20}
& treePrune& True& True& True& True& False& False& False& False& False& True& True& False& False& True& False& False& True& False\\ \cline{2-20}
\hline
\multirow{4}{*}{CART}
& threshold& 0.5& 0.01& 0.13& 0.01& 0.01& 0.06& 0.34& 0.01& 0.15& 0.39& 0.01& 0.05& 0.06& 0.01& 0.01& 0.18& 0.06& 0.01\\ \cline{2-20}
& max\_feature& None& 0.01& 0.9& 0.01& 0.01& 0.45& 0.14& 0.52& 0.57& 0.76& 1& 0.68& 0.01& 0.33& 0.01& 0.73& 0.04& 0.01\\ \cline{2-20}
& min\_samples\_split& 2& 17& 20& 7& 18& 9& 11& 15& 16& 12& 11& 2& 6& 3& 13& 19& 13& 10\\ \cline{2-20}
& min\_samples\_leaf& 1& 11& 1& 9& 2& 19& 14& 6& 15& 4& 17& 19& 9& 12& 5& 4& 7& 13\\ \cline{2-20}
\hline
\multirow{6}{*}{\begin{tabular}[c]{@{}c@{}}Random \\ Forests\end{tabular}} 
& threshold& 0.5& 0.09& 0.01& 0.01& 0.01& 0.36& 0.49& 0.08& 0.01& 0.01& 0.01& 0.05& 0.01& 0.01& 0.01& 0.2& 0.15& 0.01\\ \cline{2-20}
& max\_feature& None& 0.01& 0.63& 0.01& 1& 0.19& 0.17& 0.01& 0.01& 0.95& 0.01& 0.65& 0.54& 0.01& 0.01& 0.45& 0.63& 0.2\\ \cline{2-20}
& max\_leaf\_nodes& None& 31& 43& 28& 29& 11& 48& 17& 25& 13& 25& 10& 31& 33& 47& 44& 14& 23\\ \cline{2-20}
& min\_samples\_split& 2& 2& 7& 13& 1& 9& 13& 1& 13& 6& 10& 4& 12& 19& 16& 17& 3& 13\\ \cline{2-20}
& min\_samples\_leaf& 1& 17& 18& 2& 20& 2& 18& 9& 6& 7& 11& 17& 19& 8& 17& 3& 13& 9\\ \cline{2-20}
& n\_estimators& 100& 76& 64& 62& 141& 91& 67& 57& 104& 70& 67& 56& 59& 97& 81& 140& 85& 92\\ \cline{2-20}
\hline  \end{tabular}
}
  \caption{Parameters tuned on different models over the objective of pd}
\end{figure*}





%%%%parameters for prec %%%%%%
\begin{figure*}[!ht]

\renewcommand{\baselinestretch}{0.8}
\resizebox{\textwidth}{!}{
\scriptsize
\centering
  \begin{tabular}{|c |c |c |c |c |c |c |c |c |c |c |c |c |c |c |c |c |c |c |c |}
    \hline
  \begin{tabular}[c]{@{}c@{}}Learner \\ Name\end{tabular}&Parameters  & Default &antV0&antV1&antV2&camelV0&camelV1&ivyV0&jeditV0&jeditV1&jeditV2&log4jV0&luceneV0&poiV0&poiV1&synapseV0&velocityV0&xercesV0&xercesV1\\ 
 \hline
\multirow{8}{*}{\begin{tabular}[c]{@{}c@{}}Where\\based\\ Learner\end{tabular}}
& threshold& 0.5& 0.53& 0.48& 0.41& 0.35& 0.88& 1& 1& 0.9& 0.96& 0.57& 1& 1& 0.56& 0.57& 0.8& 0.26& 0.65\\ \cline{2-20}
& infoPrune& 0.33& 0.68& 0.74& 0.31& 0.45& 0.78& 0.31& 0.53& 0.85& 0.04& 0.73& 0.54& 0.15& 1& 0.98& 0.23& 0.9& 0.19\\ \cline{2-20}
& min\_sample\_size& 4& 3& 6& 5& 2& 1& 5& 7& 1& 7& 5& 2& 8& 1& 1& 1& 2& 2\\ \cline{2-20}
& min\_Size& 0.5& 0.07& 0.23& 0.2& 0.21& 0.25& 0.54& 0.18& 0.36& 0.28& 0.51& 1.0& 0.33& 0.5& 0.84& 0.83& 0.04& 0.21\\ \cline{2-20}
& wriggle& 0.2& 0.91& 0.77& 0.58& 0.85& 0.17& 0.66& 0.33& 0.88& 0.13& 0.72& 0.18& 0.07& 0.43& 0.63& 0.74& 0.18& 0.75\\ \cline{2-20}
& depthMin& 2& 4& 2& 5& 2& 4& 2& 2& 2& 4& 2& 3& 4& 1& 5& 1& 2& 1\\ \cline{2-20}
& depthMax& 10& 10& 14& 6& 13& 7& 16& 15& 14& 13& 19& 16& 9& 16& 14& 11& 13& 15\\ \cline{2-20}
& wherePrune& False& True& False& True& True& True& True& True& True& True& True& True& True& True& True& False& True& False\\ \cline{2-20}
& treePrune& True& True& True& True& True& True& True& False& False& False& True& False& True& True& False& False& False& False\\ \cline{2-20}
\hline
\multirow{4}{*}{CART}
& threshold& 0.5& 0.76& 0.99& 0.86& 0.48& 1& 1& 1& 0.71& 0.62& 0.65& 1& 0.95& 0.64& 0.5& 1& 0.99& 1\\ \cline{2-20}
& max\_feature& None& 0.09& 0.13& 0.05& 0.01& 0.01& 0.47& 0.01& 0.1& 0.63& 0.62& 0.44& 0.27& 0.28& 0.04& 0.75& 0.96& 0.21\\ \cline{2-20}
& min\_samples\_split& 2& 4& 19& 9& 9& 17& 14& 14& 16& 14& 12& 17& 3& 10& 12& 20& 13& 6\\ \cline{2-20}
& min\_samples\_leaf& 1& 15& 17& 8& 7& 1& 20& 1& 9& 12& 15& 10& 13& 7& 10& 4& 7& 20\\ \cline{2-20}
\hline
\multirow{6}{*}{\begin{tabular}[c]{@{}c@{}}Random \\ Forests\end{tabular}} 
& threshold& 0.5& 0.92& 0.99& 0.71& 0.7& 1& 0.82& 1& 1& 1& 0.96& 0.73& 1& 0.76& 0.33& 1& 1& 0.98\\ \cline{2-20}
& max\_feature& None& 0.23& 0.13& 0.46& 0.69& 0.37& 0.56& 0.71& 0.01& 1& 0.01& 0.85& 0.48& 0.34& 0.02& 0.01& 0.08& 0.62\\ \cline{2-20}
& max\_leaf\_nodes& None& 12& 49& 23& 39& 10& 17& 20& 10& 10& 44& 37& 18& 35& 11& 31& 35& 28\\ \cline{2-20}
& min\_samples\_split& 2& 18& 8& 14& 1& 11& 3& 2& 1& 4& 14& 2& 1& 13& 8& 18& 2& 13\\ \cline{2-20}
& min\_samples\_leaf& 1& 11& 5& 11& 3& 2& 4& 20& 2& 2& 10& 3& 7& 7& 15& 6& 2& 9\\ \cline{2-20}
& n\_estimators& 100& 130& 146& 66& 96& 50& 50& 136& 84& 83& 129& 51& 150& 99& 58& 88& 85& 61\\ \cline{2-20}
\hline  \end{tabular}
}
  \caption{Parameters tuned on different models over the objective of prec}
\end{figure*}


%%%%parameters for F %%%%%%
\begin{figure*}[!ht]

\renewcommand{\baselinestretch}{0.8}
\resizebox{\textwidth}{!}{
\scriptsize
\centering
  \begin{tabular}{|c |c |c |c |c |c |c |c |c |c |c |c |c |c |c |c |c |c |c |c |}
    \hline
  \begin{tabular}[c]{@{}c@{}}Learner \\ Name\end{tabular}&Parameters  & Default &antV0&antV1&antV2&camelV0&camelV1&ivyV0&jeditV0&jeditV1&jeditV2&log4jV0&luceneV0&poiV0&poiV1&synapseV0&velocityV0&xercesV0&xercesV1\\ 
 \hline
\multirow{8}{*}{\begin{tabular}[c]{@{}c@{}}Where\\based\\ Learner\end{tabular}}
& threshold& 0.5& 0.12& 0.78& 0.3& 0.01& 0.78& 1& 0.99& 0.44& 0.72& 0.21& 0.41& 1& 0.04& 0.7& 0.36& 0.66& 0.42\\ \cline{2-20}
& infoPrune& 0.33& 0.58& 0.2& 0.41& 0.19& 0.82& 0.91& 0.35& 1& 0.85& 0.46& 0.24& 1.0& 0.85& 0.36& 0.48& 0.42& 0.82\\ \cline{2-20}
& min\_sample\_size& 4& 6& 1& 1& 6& 8& 6& 7& 5& 1& 5& 8& 2& 7& 5& 3& 1& 1\\ \cline{2-20}
& min\_Size& 0.5& 0.8& 0.75& 0.47& 0.01& 1& 0.64& 0.99& 0.43& 0.23& 0.47& 0.72& 1& 0.89& 0.69& 0.88& 0.38& 0.75\\ \cline{2-20}
& wriggle& 0.2& 0.21& 0.7& 0.83& 0.25& 0.55& 0.01& 0.63& 0.93& 0.43& 0.33& 0.52& 0.32& 0.72& 0.1& 0.43& 0.34& 0.1\\ \cline{2-20}
& depthMin& 2& 4& 5& 4& 3& 1& 6& 1& 4& 1& 1& 4& 1& 2& 2& 4& 5& 1\\ \cline{2-20}
& depthMax& 10& 16& 11& 5& 19& 8& 10& 14& 19& 5& 6& 6& 16& 3& 11& 5& 18& 12\\ \cline{2-20}
& wherePrune& False& True& True& True& False& True& False& False& True& True& True& False& True& True& False& True& False& True\\ \cline{2-20}
& treePrune& True& False& True& True& True& True& False& False& True& False& False& False& True& True& False& False& True& True\\ \cline{2-20}
\hline
\multirow{4}{*}{CART}
& threshold& 0.5& 0.01& 0.62& 0.13& 0.01& 1& 0.8& 0.7& 0.66& 0.72& 0.32& 0.09& 0.7& 0.01& 0.01& 0.91& 0.84& 0.01\\ \cline{2-20}
& max\_feature& None& 0.24& 0.88& 0.19& 0.01& 0.01& 0.8& 0.76& 0.28& 0.5& 0.22& 0.18& 0.01& 0.58& 0.01& 0.01& 0.3& 0.01\\ \cline{2-20}
& min\_samples\_split& 2& 19& 3& 13& 5& 2& 10& 8& 9& 11& 12& 7& 10& 20& 5& 4& 14& 11\\ \cline{2-20}
& min\_samples\_leaf& 1& 15& 18& 11& 13& 3& 13& 9& 10& 15& 8& 15& 15& 7& 10& 10& 4& 1\\ \cline{2-20}
\hline
\multirow{6}{*}{\begin{tabular}[c]{@{}c@{}}Random \\ Forests\end{tabular}} 
& threshold& 0.5& 0.01& 0.49& 0.14& 0.01& 1& 1& 1& 1& 1& 0.66& 0.53& 1& 0.01& 0.21& 1& 1& 0.01\\ \cline{2-20}
& max\_feature& None& 0.89& 0.21& 0.01& 0.04& 0.81& 0.45& 0.01& 0.49& 0.01& 0.01& 0.12& 0.81& 0.01& 0.07& 0.01& 0.62& 0.61\\ \cline{2-20}
& max\_leaf\_nodes& None& 21& 16& 49& 10& 26& 32& 39& 22& 10& 42& 24& 43& 38& 10& 10& 10& 20\\ \cline{2-20}
& min\_samples\_split& 2& 11& 9& 19& 15& 6& 14& 6& 3& 4& 10& 18& 2& 8& 7& 10& 10& 16\\ \cline{2-20}
& min\_samples\_leaf& 1& 6& 11& 13& 7& 16& 4& 2& 18& 2& 6& 19& 2& 9& 4& 17& 2& 5\\ \cline{2-20}
& n\_estimators& 100& 88& 99& 124& 148& 56& 101& 116& 55& 122& 112& 75& 55& 92& 129& 58& 107& 121\\ \cline{2-20}
\hline  \end{tabular}
}
  \caption{Parameters tuned on different models over the objective of F}
\end{figure*}





%%%%%%%%%%%%%%%% defectives and non-defectives for each datasets%%%%%%%%%%%%%%%%%%%%%

\begin{figure*}[!ht]

\renewcommand{\baselinestretch}{0.8}
\scriptsize
\centering
  \begin{tabular}{c c c c c c c c c c }
  \hline\hline
  Dataset &ant&antV1&antV2&camel&camelV1&ivy&jedit&jeditV1&jeditV2
\\\hline
  training &20/125 &40/178 &32/293 &13/339 &216/608 &63/111 &90/272 &75/306 &79/312
\\  tunning  &40/178 &32/293 &92/351 &216/608 &145/872 &16/241 &75/306 &79/312 &48/367
\\  testing &32/293 &92/351 &166/745 &145/872 &188/965 &40/352 &79/312 &48/367 &11/492
\\  \end{tabular}
   \caption{The percentage of defective instances in each experimental data set. For each experiment,  training, tuning and testing data are composed of single chronological data file}
\end{figure*}
\begin{figure*}[!ht]
\scriptsize
\centering
  \begin{tabular}{c c c c c c c c c c }
  \hline\hline
  Dataset &log4j&lucene&poi&poiV1&synapse&velocity&xerces&xercesV1
\\\hline
  training &34/135 &91/195 &141/237 &37/314 &16/157 &147/196 &77/162 &71/440
\\  tunning  &37/109 &144/247 &37/314 &248/385 &60/222 &142/214 &71/440 &69/453
\\  testing &189/205 &203/340 &248/385 &281/442 &86/256 &78/229 &69/453 &437/588
\\  \end{tabular}

   \caption{The percentage of defective instances in each experimental data set. For each experiment,  training, tuning and testing data are composed of single chronological data file}
\end{figure*}


%\begin{figure*}[!th]
%  \scriptsize
%  \centering
%	\begin{tabular}{|l|l|l|}
%	\hline
%	\multicolumn{1}{|c|}{Data Set} & \multicolumn{1}{c|}{\begin{tabular}[c]{@{}c@{}} Feature Selections\\ with  Default Parameters\end{tabular}} & \multicolumn{1}{c|}{\begin{tabular}[c]{@{}c@{}}Feature Selection\\ with Tuning Parameters\end{tabular}} \\ \hline
%	  ant  & mfa, lcom3, cam, ic, dam   & wmc, lcom3, cam, dam, npm, rfc  \\ \hline
%	camel & cam, wmc, dit, mfa, rfc, loc & rfc, cam, loc, wmc, dam, dit, mfa, lcom3, npm\\ \hline
%	ivy & cam, dit, dam, ic, lcom3& cam, dam, mfa\\ \hline
%	jedit& mfa, dam, cam, dit, cbm, ic& mfa, loc, dit, dam, cam, wmc  \\ \hline
%	log4j & lcom3, mfa, loc, ic, dit& mfa, wmc  \\ \hline
%	lucene & mfa, lcom3, cam, dam, ic& wmc, lcom3, avg\_cc, rfc, ce, mfa, dam, npm, cam \\ \hline
%	poi & mfa, dam, amc, lcom3, cbm, loc& loc, lcom3, cam, amc, dam                                                     \\ \hline
%	synapse & dam, loc, mfa, cam & loc  \\ \hline
%	velocity& dit, dam, lcom3, ic, mfa& mfa, cam, avg\_cc, loc, lcom3, cbo, dam, ic, rfc, lcom \\ \hline
%	xalan  & mfa, cam, wmc, lcom3, rfc, dit & mfa, rfc, loc, dam, lcom3, wmc, npm \\ \hline
%	xerces & cam, wmc, mfa, rfc, amc, loc& mfa, loc, cam, amc, avg\_cc\\ \hline
%	\end{tabular}
%	\caption{Features selection before and after tuning over objective of G values }
%\end{figure*}


%%%%%%%%%%%%%%%% feature selection over pd%%%%%%%%%%%%%%%%%%%%%

\setlength{\tabcolsep}{3pt}
\renewcommand\arraystretch{1.2}


%%%%%%%%%%%%%% pd bars%%%%%%%%%%%%%%%%%%%%%%%%%%%%%%%% 

\begin{figure*}
\renewcommand{\baselinestretch}{0.8}
\scriptsize
\begin{minipage}{0.81\linewidth}
\begin{tabular}{r@{~}|r@{~}l@{~}|r@{~}l@{~}|r@{~}l|r@{~}@{~}l|r@{~}l@{~}|r@{~}l@{~}|r@{~}l}
  \multicolumn{1}{c|}{~}&\multicolumn{11}{c}{median(pd) } \\
  Data set   &   \multicolumn{2}{c}{Naive\_Where}         &   \multicolumn{2}{c}{Tuned\_Where}         &   \multicolumn{2}{c}{Naive\_CART}         &   \multicolumn{2}{c}{Tuned\_CART}    &   \multicolumn{2}{c}{Naive\_RanFst}  &   \multicolumn{2}{c}{Tuned\_RanFst}\\\hline
\multicolumn{1}{c}{~}\\
antV0 & 53 & {\rone} & 100 & {\rfour} & 38 &         & 100 & {\rfour} & 78 & {\rthree} & 97 & {\rfour}\\
antV1 & 7 &         & 100 & {\rfour} & 39 & {\rone} & 100 & {\rfour} & 95 & {\rfour} & 100 & {\rfour}\\
antV2 & 0 &         & 100 & {\rfour} & 37 & {\rone} & 100 & {\rfour} & 92 & {\rfour} & 100 & {\rfour}\\
camelV0 & 0 &         & 100 & {\rfour} & 6 &         & 100 & {\rfour} & 66 & {\rthree} & 99 & {\rfour}\\
camelV1 & 80 & {\rthree} & 100 & {\rfour} & 46 &         & 100 & {\rfour} & 80 & {\rthree} & 100 & {\rfour}\\
ivyV0 & 93 & {\rtwo} & 100 & {\rfour} & 88 &         & 100 & {\rfour} & 95 & {\rtwo} & 100 & {\rfour}\\
jeditV0 & 89 & {\rthree} & 100 & {\rfour} & 66 &         & 95 & {\rfour} & 96 & {\rfour} & 100 & {\rfour}\\
jeditV1 & 75 & {\rtwo} & 100 & {\rfour} & 50 &         & 100 & {\rfour} & 100 & {\rfour} & 100 & {\rfour}\\
jeditV2 & 45 &         & 100 & {\rfour} & 36 &         & 100 & {\rfour} & 100 & {\rfour} & 100 & {\rfour}\\
log4jV0 & 46 & {\rone} & 100 & {\rfour} & 31 &         & 100 & {\rfour} & 88 & {\rfour} & 100 & {\rfour}\\
luceneV0 & 81 & {\rthree} & 100 & {\rfour} & 48 &         & 100 & {\rfour} & 98 & {\rfour} & 100 & {\rfour}\\
poiV0 & 89 & {\rtwo} & 100 & {\rfour} & 79 &         & 100 & {\rfour} & 100 & {\rfour} & 100 & {\rfour}\\
poiV1 & 2 &         & 100 & {\rfour} & 12 &         & 100 & {\rfour} & 89 & {\rfour} & 100 & {\rfour}\\
synapseV0 & 0 &         & 100 & {\rfour} & 28 & {\rone} & 100 & {\rfour} & 90 & {\rfour} & 100 & {\rfour}\\
velocityV0 & 100 & {\rfour} & 100 & {\rfour} & 86 &         & 100 & {\rfour} & 100 & {\rfour} & 100 & {\rfour}\\
xercesV0 & 64 & {\rone} & 100 & {\rfour} & 46 &         & 100 & {\rfour} & 78 & {\rtwo} & 91 & {\rfour}\\
xercesV1 & 15 &         & 100 & {\rfour} & 11 &         & 100 & {\rfour} & 68 & {\rthree} & 87 & {\rfour}\\
%Albrecht  &   7         &   24   &   Projects from IBM   & 28 &     & 40 & {\rtwo} & 38 & {\rtwo} & 38 & {\rtwo} & 49 & {\rfour} \\
%China      &   18       &   488   &   Projects from Chinese software companies   &   38   &  {\rtwo}  &   34   &       &   34   &       &   35   &  {\rone}   &   41   &   {\rfour}\\
%Cosmic     &   y   &   y   &   Projects described in functiion points   &   98   &   {\rfour}   &   75   &       &   85   &    {\rtwo}   &   85   &   {\rtwo}    &   89   &   {\rtwo}\\
%ISBSG10    &   y   &   y   &   From the ISBSG benchmark suite   &   56   &       &   62   &     &   126   &   {\rfour}   &   66   &   {\rone}   &   65   &   {\rone}\\
%Kemerer    &   7   &   15   &   Large business applications   &   42   &  {\rtwo}  &   24   &   &   55   &  {\rfour} &   55   &   {\rfour}   &   55   &   {\rfour}\\
%Kitchenham &    6   &   145   &  y    &   34   &   &   43   &   {\rthree}   &   34   &  &   43   &  {\rthree}  &   47   &  {\rfour}  \\
%Maxwell    &   27  &   62  & Projects from commercial banks in Finland   &   57   &   {\rtwo}   &   56   &   {\rtwo}   &   47   &   &   53   &  {\rone} &   64   &   {\rfour}\\
%Miyazaki   &   8   &   48 &Japanese software projects developed in COBOL   &   39   &       &   41   &   &   41   &    &   57   &   {\rfour}   &   57   &   {\rfour}\\
%Telecom    &   3   &   18   &   Maintenance projects for telecom companies   &   23   &       &   26   &   {\rtwo}   &   31   & {\rfour}  &   31   &   {\rfour}   &   31   &   {\rfour}\\
%Usp05      &   7   &  203  &  Collected from university student projects   &   30   &    &   50   & {\rfour} &   45   & {\rthree}  &   40   &   {\rtwo}  &   50   &   {\rfour}\\
\end{tabular}
\end{minipage}\begin{minipage}{.15\linewidth}
\begin{tabular}{|p{\linewidth}|}\hline

~\\

{\bf KEY:}

~\\

pd percentile ranges:

~\\

80th to 100th = {\rfour}

60th to 80th ~ = {\rthree}

40th to 60th  ~ = {\rtwo}

20th to 40th  ~ = {\rone}

~\\

An absent bar denotes\newline 0th to 20th percentile.

~\\

Percentiles computed  separately
for each data set.\\\hline
\end{tabular}
\end{minipage}
\caption{Median pd values in tune once and test ten times experiment. 
Gray bars  show  pd values
discretized into 20th percentiles ranges from min to max.
All data available from http://openscience.us/repo/effort.
}\label{fig:nonc}
\end{figure*}

%%%%%%%%%%%%%% pd bars end%%%%%%%%%%%%%%%%%%%%%%%%%%%%%%%% 


%%%%%%%%%%%%%% precision bars%%%%%%%%%%%%%%%%%%%%%%%%%%%%%%%% 

\begin{figure}
\renewcommand{\baselinestretch}{0.8} 

\scriptsize  
\begin{tabular}{r|r@{~}l@{~}|r@{~}l|r@{~}l|r@{~}l|r@{~}l@{~}|r@{~}l@{~}r@{~}l}
      &   \multicolumn{4}{c|}{WHERE}         &   \multicolumn{4}{c|}{CART}         &   \multicolumn{4}{c}{Random Forests}         \\\hline
  Data set   &   \multicolumn{2}{c}{Naive}         &   \multicolumn{2}{c|}{Tuned}         &   \multicolumn{2}{c}{Naive}         &   \multicolumn{2}{c|}{Tuned}    &   \multicolumn{2}{c}{Naive}  &   \multicolumn{2}{c}{Tuned}\\\hline
 
antV0 & 30 &         & $\bigstar$ 89 & {\rfour} & 27 &         &$\bigstar$ 89 & {\rfour} & 40 & {\rone} &$\bigstar$ 89 & {\rfour}\\
antV1 & 32 &         &$\bigstar$ 74 & {\rfour} & 41 & {\rone} &$\bigstar$ 74 & {\rfour} & 57 & {\rtwo} &$\bigstar$ 74 & {\rfour}\\
antV2 & 78 & {\rfour} &$\bigstar$ 78 & {\rfour} & 52 &         & 67 & {\rthree} & 66 & {\rtwo} & 50 &        \\
camelV0 &$\bigstar$ 83 & {\rfour} &$\bigstar$ 83 & {\rfour} & 26 &         & 37 &         &$\bigstar$ 83 & {\rfour} &$\bigstar$ 83 & {\rfour}\\
camelV1 & 22 &         &$\bigstar$ 81 & {\rfour} & 23 &         & 25 &         & 28 &         & 28 &        \\
ivyV0 & 16 &         & 23 & {\rthree} & 18 & {\rone} &$\bigstar$ 25 & {\rfour} & 18 & {\rone} & 19 & {\rone}\\
jeditV0 & 35 &         & 75 & {\rthree} & 49 & {\rone} &$\bigstar$ 86 & {\rfour} & 52 & {\rone} & 50 & {\rone}\\
jeditV1 & 24 &         &$\bigstar$ 87 & {\rfour} & 28 &         & 62 & {\rthree} & 36 &         & 42 & {\rone}\\
jeditV2 & 2 &         &$\bigstar$ 98 & {\rfour} & 3 &         & 4 &         & 5 &         & 6 &        \\
log4jV0 & 94 &         &$\bigstar$100 & {\rfour} & 97 & {\rtwo} & 98 & {\rthree} &$\bigstar$ 100 & {\rfour} &$\bigstar$100 & {\rfour}\\
luceneV0 & 61 &         & 71 & {\rtwo} & 67 & {\rone} &$\bigstar$ 78 & {\rfour} & 69 & {\rtwo} & 70 & {\rtwo}\\
poiV0 & 70 &         &$\bigstar$ 92 & {\rfour} & 77 & {\rone} & 79 & {\rtwo} & 79 & {\rtwo} & 75 & {\rone}\\
poiV1 & 100 & {\rfour} &$\bigstar$ 89 & {\rfour} & 73 & {\rtwo} &$\bigstar$ 89 & {\rfour} & 86 & {\rthree} & 36 &        \\
synapseV0 & 66 & {\rthree} & 0 &         & 71 & {\rthree} &$\bigstar$ 95 & {\rfour} & 59 & {\rthree} & 67 & {\rthree}\\
velocityV0 & 34 &         & 34 &         & 34 &         &$\bigstar$ 45 & {\rfour} & 40 & {\rtwo} & 41 & {\rthree}\\
xercesV0 & 13 &         &$\bigstar$ 85 & {\rfour} & 14 &         & 73 & {\rfour} & 16 &         & 13 &        \\
xercesV1 &$\bigstar$ 56 & {\rfour} & 26 &         & 55 & {\rfour} & 26 &         & 41 & {\rtwo} & 26 &        \\
\end{tabular}

\caption{Precision values in tune and naive runs.
All data available from http://openscience.us/repo/effort. {\bf KEY:}
  percentile ranges:\newline
80th to 100th= {\rfour};
60th to 80th= {\rthree}; 
40th to 60th= {\rtwo};
20th to 40th= {\rone};
an absent bar shows  0th to 20th.
Percentiles computed  separately
for each row.

}\label{fig:nonc}
\end{figure}

%%%%%%%%%%%%%% precision bars end%%%%%%%%%%%%%%%%%%%%%%%%%%%%%%%% 

\begin{figure*}
\renewcommand{\baselinestretch}{0.8}
\scriptsize
\begin{minipage}{0.81\linewidth}
\begin{tabular}{r@{~}|r@{~}l|r@{~}l@{~}|r@{~}l|r@{~}@{~}l|@{~}r@{~}l|r@{~}l|@{~}r@{~}l}
  \multicolumn{1}{c|}{~}&\multicolumn{11}{c}{median(F ) } \\
  Data set   &   \multicolumn{2}{c}{Naive\_Where}         &   \multicolumn{2}{c}{Tuned\_Where}         &   \multicolumn{2}{c}{Naive\_CART}         &   \multicolumn{2}{c}{Tuned\_CART}    &   \multicolumn{2}{c}{Naive\_RanFst}  &   \multicolumn{2}{c}{Tuned\_RanFst}\\\hline
\multicolumn{1}{c}{~}\\
ant0 &$\bigstar$ 39 & {\rfour} & 26 & {\rone} & 32 & {\rtwo} & 25 &         & 25 &         & 22 &        \\
ant1 & 11 &         &$\bigstar$ 85 & {\rfour} & 40 & {\rone} & 49 & {\rtwo} & 39 & {\rone} & 33 & {\rone}\\
ant2 & 0 &         &$\bigstar$ 86 & {\rfour} & 44 & {\rtwo} & 51 & {\rtwo} & 52 & {\rthree} & 56 & {\rthree}\\
camel0 & 0 &         &$\bigstar$ 28 & {\rfour} & 9 & {\rone} &$\bigstar$ 28 & {\rfour} & 34 & {\rfour} & 31 & {\rfour}\\
camel1 & 34 & {\rthree} &$\bigstar$ 35 & {\rfour} & 31 &         & 33 & {\rtwo} & 33 & {\rtwo} & 30 &        \\
ivy0 & 27 &         & 30 & {\rone} & 30 & {\rone} & 32 & {\rthree} &$\bigstar$ 35 & {\rfour} &$\bigstar$ 35 & {\rfour}\\
jedit0 & 50 &         & 55 & {\rtwo} & 56 & {\rtwo} & 55 & {\rtwo} &$\bigstar$ 61 & {\rfour} & 60 & {\rfour}\\
jedit1 & 37 & {\rone} & 34 &         & 36 &         & 47 & {\rfour} & 45 & {\rthree} &$\bigstar$ 48 & {\rfour}\\
jedit2 & 4 &         &$\bigstar$ 99 & {\rfour} & 5 &         & 9 &         & 9 &         & 11 &        \\
log4j0 & 62 & {\rfour} & 7 &         & 47 & {\rthree} &$\bigstar$ 64 & {\rfour} & 53 & {\rfour} & 45 & {\rthree}\\
lucene0 & 70 & {\rthree} & 73 & {\rfour} & 56 &         &$\bigstar$ 75 & {\rfour} & 70 & {\rthree} &$\bigstar$ 75 & {\rfour}\\
poi0 &$\bigstar$ 78 & {\rfour} & 63 &         & 74 & {\rthree} & 70 & {\rtwo} & 73 & {\rthree} & 72 & {\rtwo}\\
poi1 & 5 &         &$\bigstar$ 78 & {\rfour} & 21 & {\rone} &$\bigstar$ 78 & {\rfour} & 76 & {\rfour} &$\bigstar$ 78 & {\rfour}\\
synapse0 & 0 &         & 2 &         & 40 & {\rthree} &$\bigstar$ 56 & {\rfour} & 52 & {\rfour} & 55 & {\rfour}\\
velocity0 & 51 &         & 51 &         & 49 &         & 51 &         & 53 & {\rone} &$\bigstar$ 59 & {\rfour}\\
xerces0 & 22 & {\rthree} & 20 & {\rone} & 21 & {\rtwo} & 18 &         & 23 & {\rfour} & 22 & {\rthree}\\
xerces1 & 23 & {\rone} & 2 &         & 18 & {\rone} & 36 & {\rtwo} & 68 & {\rfour} &$\bigstar$ 71 & {\rfour}\\
\end{tabular}
\end{minipage}\begin{minipage}{.15\linewidth}
\begin{tabular}{|p{\linewidth}|}\hline

~\\

{\bf KEY:}

~\\

F percentile ranges:

~\\

80th to 100th = {\rfour}

60th to 80th ~ = {\rthree}

40th to 60th  ~ = {\rtwo}

20th to 40th  ~ = {\rone}

~\\

An absent bar denotes\newline 0th to 20th percentile.

~\\

Percentiles computed  separately
for each data set.\\\hline
\end{tabular}
\end{minipage}
\caption{Median F values in tune once and test ten times experiment. 
Gray bars  show  F values
discretized into 20th percentiles ranges from min to max.
All data available from http://openscience.us/repo/effort.
}\label{fig:nonc}
\end{figure*}

%%%%%%%%%%%%%% F bars end%%%%%%%%%%%%%%%%%%%%%%%%%%%%%%%% 


%%%%%%%%%%%%%%%%  counts of features for different goals%%%%%%%%%%%%%%%%%%%%%
\begin{figure}[!ht]

\renewcommand{\baselinestretch}{0.8}
\scriptsize
\centering
  \begin{tabular}{c|c c|c c|c c|c c| c c }
  
    & \multicolumn{2}{c|}{Pd} &  \multicolumn{2}{c|}{Precision} & \multicolumn{2}{c|}{F} &  \multicolumn{2}{c|}{SUM}\\
 &&&&&&&&\\
Features& \begin{sideways}naive\end{sideways}
& \begin{sideways}tuned\end{sideways}
& \begin{sideways}naive\end{sideways}
& \begin{sideways}tuned\end{sideways}
& \begin{sideways}naive\end{sideways}
& \begin{sideways}tuned\end{sideways}
& \begin{sideways}naive\end{sideways}
& \begin{sideways}tuned\end{sideways}
\\\hline
noc&	&	&	&	&	&	&	&	&	\\
ca&	&	&	&	1&	&	&	&	1&	\\
max\_cc&	&	&	&	1&	&	&	&	1&	\\
moa&	&	&	&	1&	&	1&	&	2&	\\
avg\_cc&	&	&	&	3&	&	2&	&	5&	\\
cbo&	&	&	&	1&	&	3&	&	4&	\\
npm&	&	&	&	2&	&	4&	&	6&	\\
lcom&	&	&	&	1&	&	4&	&	5&	\\
ce&	&	&	&	3&	&	2&	&	5&	\\
amc&	4&	&	4&	1&	4&	5&	12&	6&	\\
cbm&	6&	&	4&	1&	4&	2&	14&	3&	\\
rfc&	4&	&	4&	4&	4&	9&	12&	13&	\\
wmc&	5&	&	5&	3&	5&	7&	15&	10&	\\
ic&	8&	1&	9&	3&	8&	8&	25&	12&	\\
dit&	8&	1&	8&	5&	7&	8&	23&	14&	\\
cam&	9&	&	9&	3&	9&	8&	27&	11&	\\
loc&	9&	1&	9&	4&	9&	8&	27&	13&	\\
lcom3&	9&	&	8&	5&	8&	13&	25&	18&	\\
dam&	14&	&	14&	6&	14&	12&	42&	18&	\\
mfa&	16&	3&	16&	6&	16&	16&	48&	25&	\\

  \end{tabular}
    \caption{Counts of features selected by different goals. For each goal, the numbers in right and left columns represent the counts of features selected for all the data sets with and without tuning processes.}
\end{figure}


%%%%time for G %%%%%%

\section{Discussion}
Data torturing, on not? Offer insight on the complexities of data sets, demanding
that before they are used, we spend more time describing the context of use. Not offering
conclusions that are over-stated. Taking care to precisely describe the boundaries of a 
conclusion. While at the same time showing how to makde better conclusions on new
data, whenever it arrives. And ensuring that those new conclusions are tuned
to the goals of the business users who are funding the analysis and will use its results.





\begin{figure}[!t]
\small
\begin{tabular}{|p{.95\linewidth}|}\hline
Some goals related to aspects of defect prediction:
\be
\item
Mission-critical systems are risk averse and may accept very high false alarm rates,
just as long as they catch any life-threatening possibility. That is, such projects
do not care about effort- they want to {\em maximize recall} regardless of any impact
that might have on the false alarm rate.
\item
Cost averse managers may accept lower probabilities of
detection (the recall measure), just as long as they {\em do not waste budgets on false alarms}. This community
seeks to {\em minimize false alarms} while maintaining {\em some level of adequate recall}.
\item  Suppose a new hire wants
 to impress their manager. That 
 new hire might want to ensure that no result presented to  management contains  true negative;
i.e. they wish to {\em maximize precision}.
\item
Some communities do not care about   low precision,
just as long as a small fraction the data is returned. Hayes, Dekhytar, \& Sundaram call this fraction 
{\em selectivity} and offer an
extensive discussion of the merits of this measure~\cite{hayes06}.
\ee
\\\hline
Beyond defect prediction are other goals that combine defect prediction with other economic
factors:
\be
\setcounter{enumi}{4}
\item
Arisholm~\&~Briand~\cite{arisholm06},  Ostrand \& Weyeuker~\cite{ostrand04} and Rahman et al.~\cite{rahman12}
say that a defect predictor should maximizing {\em reward}; i.e. find the fewest lines of code
that contain the most bugs.
\item In other work, Yin et al. are concerned about
 {\em incorrect bug fixes}; i.e. those that require subsequent work in order to complete the bug fix.
These bugs occur  when (say) developers try to fix parts of the code
where they have very little experience~\cite{yin11}.  To avoid such incorrect bug fixes, we have to optimize
for finding the most number of bugs in regions that {\em the most programmers have worked with before}.
\item In {\em Better-faster-cheaper}, we seek  project changes that lead
to fewer defects and faster development times using less resources~\cite{Green,elrawas08,elrawas10,me07f,me09a,me09f}.
\item {\em  Rush-to-market} is another economic-based optimization measure.
A learner that tries to maximize ``rush-to-market'' is trying to release the product as soon
as possible, without too many bugs. Note that ``rush-to-market'' is an appropriate strategy for a company competing
in a volatile and crowded market place where being first-to-market enables a revenue stream (that can be
used to subsequently fix any issues with version 1.0)~\cite{huang06}.
\ee
\\\hline
All the above measures relate to the tendency of a predictor to find something. Another style
of measure would be to check the {\em variability} of that predictor:
\be
\setcounter{enumi}{8}
\item
In their study on reproducibility of SE results,
 Anda, Sjoberg and Mockus advocate using the coefficient of variation ($CV=\frac{stddev}{mean}$).
Using this measure, they defined {\em reproducibility} as $\frac{1}{CV}$~\cite{anda09}.
\ee
\\\hline
\end{tabular}
\caption{Reasoning about software many have many different goals including the nine shown here. Many of
these goals are defined in terms of 
\fig{criteria}.
}\label{fig:goals}
\end{figure}
\fig{goals} lists other performance measures that we have seen at client sites or in the literature. Note that:
\be
\item This list is very long.
\item This list keeps growing.
\ee As to this second point,
often when we work
with new clients, they surprise us with yet another domain-specific criteria that is important for the business.
That is, neither \fig{criteria} or \fig{goals} is not a complete list of all possible assessment criteria.

Our reading of the defect prediction literature is that most papers only explore a small subset of 
 \fig{criteria} or \fig{goals}. We think this is a mistake and researchers should more to a more general
framework where they explore a wide and changing set of evaluation criteria. 
Hence, this paper.



\end{document}

O 
PEEKING2's PCA tool used an accelerated   principle component analysis that synthesises  new
attributes $e_i, e_2,...$
that extends across the dimension of greatest  variance in the data  with attributes $d$.  
PCA  combines
redundant  variables into a smaller set of variables  (so $e \ll d$) since those
redundancies become (approximately) parallel lines
in $e$ space. For all such redundancies \mbox{$i,j \in d$}, we 
can ignore $j$ 
since effects that change over $j$ also
change in the same way over $i$.
PCA is also useful for skipping over noisy variables from $d$-- these
variables are effectively ignored since    they  do not contribute to the variance in the data.


In the case of WHERE, this reduction incluFor CART, the measure of compression is



XXX One consequence of this work is that it is possible to tune a detector such that it exhibits a wide range of performance scores. Given that, the next question becomes ``what performance scores do you want?''. This is an interesting issue since there is no single best answer. XXX

How ``true'' are the  models generated by software analytics?
Suppose we use a model generated from software project data to, for example,
assess the relative value of OO metrics vs procedural metrics for predicting
project defects. Are we reporting ``truth'' in any sense of the word? And for how
many other projects  and how many kinds of questions might we find that same ``truth''? 

This is not some abstract question-- it is an open and pressing matter.
Any reading of the recent proceedings of FSE, ICSE, ASE, etc will discover hundreds
of papers using data miners to make conclusions about software projects. 
For example, some  analysts have made conclusions
about 
{\em what attributes
have most influence on a software project}.
For example, in 2013,
Rahman et al
\cite{rahman2013how}  used results from logistic regression to conclude
that code metrics are generally less useful than process
metrics for  predicting defects from static code analysis.  
In the same year, Bell et al. (who also used negative binomial regression) argued that
software defect prediction was not improved by adding attributes relating to the social
structure of programmer teams~\cite{bell2013limited}.

@article{bell2013limited,
  title={The limited impact of individual developer data on software defect prediction},
  author={Bell, Robert M and Ostrand, Thomas J and Weyuker, Elaine J},
  journal={Empirical Software Engineering},
  volume={18},
  number={3},
  pages={478--505},
  year={2013},
  publisher={Springer}
}

The results of this paper offer a very different picture  to recent prominent results from prior authors.
Lessmann et al.~\cite{lessmann2008benchmarking}, Rahman et al.~\cite{rahman2013how} and Arshiholm et al.~\cite{arisholm06} argue that when it comes to defect prediction, different learners often
produce statistically similar results. For this reason  Rahman et al. only present results their logistic
regression results, commenting that ``dor brevity, only results from LR are
presented; other learning models give similar results''~\cite{rahman2013how}. Note that those prior results,
where most learners achieved the same results, arose from the default parameter settings from those learners.

Our results, on the other hand, come from tuning a data miner's parameters. In those results,
we found that the choice of parameters had dramatic impacts on data mining performance.
For example, we show results below where:   
\bi 
\item 
Tuning improved precision values from  2\% to 98\%; or
\item 
Tuning improved false alarm values from 69\% to 0\%; or
\item 
Tuning improved recall values from 0 to 100\%; or
\item 
Tuning improved the f-measure (which is the harmonic mean of recall and precision) from 0 to 86\%.
\ei   
In terms of changing SE research methodologies, an important aspect of these results is that usually XXXX always different. cannot determine without experimentation which learner will, after tuning, yield the best results for a particular data set. Hence it is no longer viable for researchers to trust the recommendations 
of other researchers working with other data sets that ``data miner XYZ is best''. Before
offering a conclusion about a particular data set, researchers need now augment  their 
papers with preliminary tuning studies (the results of which select the tunings+data miners to be used
to make conclusions for the second half of their paper).

The good news is that such tuning is not an arduous task. The differential evolution algorithm 
used in this study is wdiely available in many standard  toolkits and languages\footnote{See the long list at
www1.icsi.berkeley.edu/~storn/code.html.}
And even if somehow those tools and prior implementations are not workable at a local site, 
differential evolution is very simple to code. Hence, it can readily and quickly implemented.
 

When we discuss these results with colleagues, one response we get is ``enough of the parameter tuning.. lets use simpler methods''. 
While we understand their motivation (to keep thing simple), software project
data cannot necessarily be condensed to (say) a simple univariate analysis that
reports how one independent variable effects a dependent variable.
Such a simplistic analysis can miss significant interaction
effects when multiple factors combine to effect the target variable. For example,
one of the base results in defect prediction is that predictors formed from
a single variable perform much worse than those can combined multiple variables (three or more)~\cite{me07b}.

If those results come from an analysis of a complex interactions between attributes,
then some kind of data mining method\footnote{By this term we mean the
full gamut of methods for large-scale analysis of
data including statistical
parametric methods (that assume a particular distribution then strive to learn the parameters of that distribution); kernel methods;
dimensionality  synthesis methods; 
instance weighting schemes; feature subset selection methods
outlier removal methods;
decision tree learners; neural networks; and instance-based
methods including clustering and/or   case-based reasoning.} then there is the
{\em tuning problem} of how to set the myriad of parameters that 
control the learner.

\item
What data miners are better than others for particular tasks
in software analytics. 
\ei



Also, in 2007, one of us (Menzies) claimed that Naive Bayes
classifiers with a certain lind of pre-processor were better than entropy-based decision trees for defect prediction~\cite{me07b}.
Similarly, in 2008 Lessmann et al.~\cite{lessmann2008benchmarking} argued  that 
Random Forests~\cite{brieman00} and CART ~\cite{brieman84} work best, worst for learning software defect predictors from
static code attributes. Similar claims that one learner was better than another
have been made by many other authors; e.g. 
 
 


We write to suggest that all those conclusions may be wrong or, at the very most,
correct within a very limited context. At the very least, the conclusions
of those papers (and literally dozens of similar papers, include severl
of our own~\cite{me02e,me02k,me03a,me06d,me07b}) need to be revisited
in light of this paper's results on the effects of tuning on data miners.

Machine learners
are controlled by parameters that decide (e.g.) when to stop recursively
dividing data into smaller section or (e.g.) what thresholds for errors are small
enough to be acceptable. When used ``off-the-shelf'', analytst
use default settings for those parameters. We find that  a 
simple optimization scheme (differential evolution) can find much
better tunings that can dramatically change the performance of
the learned model (measured in terms of false alarm rates, recall,
precision, f-measure, and g-measure).  From this first result,
we conclude that {\bf parameter tuning with some optimizer such as differential
evolution should become standard practice}.

Our next observation is that observations about miner1 being better
than miner2 do not survive tuning. To be specific: given some
ranking of data miners for some task such as defect prediction,
we find that tuning can reverse those rankings such that the worst untu


Tuning the data miners with different evolution takes some
times (around an order of magnitude slower than the learning), but not an impractically slower amount of time. Further, when we look at what
attrub


Our reading of the 

good news:
\bi 
\item Tuning is easy (but one measure, at least an order of magnitude easier than 
standard optimization problems)
\item Tuning can dramatically improve the performance of a learner (in one case, from recalls
of zero to 70\%)
\ei
interesting news
\bi 
\item tunings different for different data sets.
\ei 
bad news:
\bi
\item Tuning also changes what was learned
\bi
\item Supposedly ``bad'' learners start working much better than supposedly ``best'' learners;
\item The models used in the tuned models were very different to those used in the untuned learners.
\ei 
\ei 
and we reflect on that model, are we 

will it reveal the `1`t
A standard approach to software analutics
Software has becoming a large and complex system and delivering reliable and quality 
software is imperative for development teams. Empirical study shows that the longer the 
defects exist in software systems, the more the cost of time and money it will take to fix it 
\textbf{ [need a ref]}. Therefore, project managers and software programers strive to find 
defects in their system as early as possible. Defect prediction has been investigated 
extensively in industrial and academia during the past two decades. As an important research 
field, building data miners  \cite{lessmann2008benchmarking, mccabe1976complexity, 
menzies2007data, menzies2010defect, jiang2008can, menzies2011local, song2011general} 
over static code features of software system has been demonstrated to be a way to predict 
which models are more likely to contain defects.

Classification is an important approach to predict whether some modules in the projects are 
defective or non-defective. The general idea is to train the learners by using parts of data 
sets(e.g. ant 1.3, 1.4 in PROMISE\footnote{http://openscience.us/repo/}) and predict with 
remaining ones(ant 1.5, 1.6 and 1.7). Many types of defect predictors have been proposed 
based an different data mining classifiers, including CART, Random Forest 
\cite{guo2004robust},  Naive Bayes\cite{menzies2007data},Logistic Regression 
\cite{khoshgoftaar1999logistic}. During the past years, authors claimed that their new 
defective learner outperformed others according to their experiment and statistical analysis. 
To evalute those learners objectively in terms of accuracy, Lessmann et al
\cite{lessmann2008benchmarking} carried out a study to compare 22 classifiers over 10 public 
\begin{figure}
\small
\begin{tabular}{|p{.95\linewidth}|}\hline
When applying case-based reasoning, researchers use different settings including how they measure dis- tance between instances (examples include the Euclidean measure, Manhattan and Chebyshev). Other settings control if numerics are normalized min..max to 0..1, before they are used in the distance measure.

Another set of settings refers to how to handle non-numeric distances or missing values. Some researchers delete training data with missing information while others compute missing values via, say, median values.

After feature selection comes instance selection. There are many settings for instance selectors such as use outlier removal or prototype generation. Within each of those settings are number sub-choices such as the choice of outlier remover or prototype generator.
The next set of settings concern how many neighbors to use (k=1,2,5,etc). Some researchers set ``k'' in dynamic pre-learning phase that checks the training data to find what ``k'' works best for that data.

Once the neighbors are known, the next settings control how to summarize their information. Some researchers use means, other use medians, or a weighted sum that uses the distance of the neighbor from the test instance (and that weighting might be determined by some kernel such as uniform, triangular, Gaussian, cosine, or Epanechnikov).
Researchers may use other settings to control different pre-processors. E.g. some discretize numeric values (via equal width or equal frequency) using some binning policy (and researchers differ on how many bins are best).\\\hline
\end{tabular}
\caption{Some tunings for SE predictive systems. Found by this author while preparing this proposal in a day of reading the SE predictive modeling literature on case-based reasoning and effort estimation. Note that other kinds of prediction systems (decision trees, neural nets, linear repression, etc) have yet more adjustable design options.}\label{fig:adjust}
\end{figure}

domain data sets from the NASA Metrics Data repository. By using Nemenyi's post hoc test 
with $\alpha = 0.05$, they concluded that the predictive accuracy of most learners didn't differ 
significantly in terms of the area under the receiver operating characteristics curve(AUC). 
Furthermore, according to the fig.2 in \cite{lessmann2008benchmarking}, Random Forest is 
significantly better than CART. Lessmann's paper motivates us to investigate whether tuning 
those CART's parameters by search-based software engineering method can improve the 
performance. Even though Lessmann considered unpruned tree and pruned tree, They didn't 
consider other possible parameters in CART which would have impact on the structure of 
trees, like the depth of the tree, the maximum and minimum number of leafs of the tree.

Software metrics are the core of all the defective prediction model. Many types of metics
are used to build models, like process metrics, McCabe and Halsted metrics  and CK metrics.
By building prediction modes across 85 releases of 12 open source projects, Rahman et al
\cite{rahman2013how}  concluded that code metrics are generally less useful than process
metrics for prediction. And also the code metrics don't change much from release to release
and lead to stagnation in the prediction model. In \cite{Radjenovi?20131397}, Radjenovi? et al
\cite{Radjenovi?20131397} reviewed 106 papers regarding software prediction metrics. They found
that CK objected-oriented and process metrics have been reported to be more successful in
finding defects compared to traditional size and complexity metrics. Moreover, not all the CK
metrics perform well equally. The best metrics from CK are CBO, WMC and RFC based on their 
observation. It seems that the relationship between software metrics and defective prediction is
still an open question and need to be addressed. This motivates us to see : whether the impact
rankings of those metrics will change after tuning parameters
is applied to model learners.



What's the problem in those result?

RQ:

briefly describe our study and  our result; observation

structure of this paper.

Note that we do not claim that differntial evolution is the {\em best} way to
optimize data miners for defect prediction. In this paper, we only show that
differential evolution is {\em simple}, {\em fast} and {\em effective} at that task. As to   finding the best optimizer for defect prediction, that task would require
the input of many researchers (and we hope this paper inspires
much subsequent work into that issue).

\subsection{Implications}

time for an end to era of data mining in se? moving on to a new phase of learning-as-optimization

1) learning is actually an optimization tasks (e.g. see fig2 of  learners climbing the roc curve hill in http://goo.gl/x2EaAm)

2) our learners are all contorted to do some tasks X (e.g. minimize expected value of entropy), then we assess them on score Y (recall). which is nuts. maybe we should build the goal predicate into the learner (e.g http://menzies.us/pdf/10which.pdf) 

3) given 1 + 2, maybe the whole paradigm of optimizing param selection is wrong. maybe what we need is a library of bees buzzing around making random choices (e.g. about descritziation) which other bees use, plus their own random choices (e.g. max depth of tree learned from discretized data) which is used by other bees, plus their own random choices (e.g. business users reading the models).  the funky thing here is that it can take some time before some of the bees (the discretizers) get feedback from the community of people using their decision (the tree learners). 




